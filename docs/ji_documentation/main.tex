\documentclass[
	fontsize=12pt,
	paper=a4,
	twoside=false,
	titlepage=true,
]{article}

% Setup TOC
\renewcommand{\contentsname}{Inhaltsverzeichnis}

% Setup Bibliography
\usepackage[
    backend=biber,
    citestyle=numeric-comp,
    bibstyle=numeric,
    sorting=none
]{biblatex}
\setlength{\bibitemsep}{0.5em}

% Setup Images
\usepackage{tocloft}
\usepackage{graphicx}
\usepackage{subcaption}
\usepackage{placeins}
\usepackage{float}
\renewcommand{\figurename}{Abbildung}

% Setup Page Styling
\usepackage[
    top=2.5cm, % Top margin
    bottom=2.5cm, % Bottom margin
    left=3cm, % Left margin
    right=3cm, % Right margin
    footskip=1cm, % Space from the bottom margin to the baseline of the footer
    headsep=0.75cm, % Space from the top margin to the baseline of the header
]{geometry}

\addbibresource{references.bib}

\title{
	Projektbericht MoleHole
\\}
\author{
	Gabriel Martin\textsuperscript{1} und Leonhard Neuhold\textsuperscript{1}
\\}
\date{
	\footnotesize\textsuperscript{\textbf{1}}IT-Abteilung, HTL Krems%
}

\begin{document}


	% #########################################################################################
	%
	%		Title Page
	%
	% #########################################################################################

	\maketitle

	\begin{abstract}
		Dies ist der Projektbericht zum MoleHole Projekt im Rahmen des Faches Projektmanagement
		der IT-Abteilung an der HTL Krems.
		Dieses Projekt beschäftigt sich mit der Visualisierung und Simulation von schwarzen Löchern.
		Dafür soll Rücksicht auf verschiedenste physikalische Effekte wie den Gravitationslinseneffekt
		oder die Schwarzschildmetrik genommen werden.
		Weiters werden für die Darstellung dieser Erscheinungen Technologien wie OpenGL und in Spezialisierung
		Ray Tracing und Ray Marching verwendet.
	\end{abstract}

	\tableofcontents


	% #########################################################################################
	%
	%		Section Introduction
	%
	% #########################################################################################

	\section{Einleitung} \label{sec:introduction}

	%! Author = leonh
%! Date = 31.12.2025

Schwarze Löcher sind komplexe Objekte.
Jedoch kann diese jeder verstehen, wenn er möchte.
Dies ist mit spielerischem Erlernen und Ausprobieren möglich.
Das ist das Konzept des MoleHole Projekts.
Dieses Dokument schildert die wichtigsten Fakten und Erkenntnisse aus
dem Projektverlauf.
Dabei wird auf die Idee der Simulation schwarzer Löcher, die anfängliche Planung und
das Projektmanagement eingegangen.
Weiters wird beschrieben, wie dieses Projekt einen bleibenden Eindruck in der Gesellschaft
hinterlassen kann und wie es die Wissenschaftskommunikation verbessern soll.
Danach werden einige technische Details erklärt, die im Hintergrund für die Visualisierung
und Simulation der schwarzen Löcher verantwortlich sind.
Es werden auch einige Grafiken zu sehen sein, die verdeutlichen sollen, wie die Applikation, die das
Ergebnis des Projekts bildet, aussieht.
Damit kann man erkennen, welche verschiedenen Möglichkeiten die Software bietet.
Ein wichtiger Teil ist außerdem der Ausblick und das Potenzial, das in diesem Projekt steckt.
Zum Abschluss wird ein Bericht über den Projektverlauf vom Projektkoordinator,
Gabriel Martin, verfasst.



	% #########################################################################################
	%
	%		Section Projektdokumentation
	%
	% #########################################################################################

	\section{Projektdokumentation} \label{sec:project_documentation}

	%! Author = leonh
%! Date = 31.12.2025

In diesem Teil des Berichtes findet man die Dokumentation zum MoleHole Projekt.
Es sind alle wichtigen Informationen und Erkenntnisse aus dem Projektverlauf zusammengefasst.
Es soll ein möglichst klares Bild der entstandenen Applikation vermittelt werden.
	
	%! Author = leonh
%! Date = 31.12.2025

\subsection{Projektentstehung und -planung} \label{subsec:project_origin_and_planning}

\begin{figure}[b]
    \centering
    \includegraphics[width=0.4\textwidth]{}
    \caption{Originale Notizen bei der Aufstellung der Projektziele}
    \label{fig:original_notes_project_goals}
\end{figure}

\textbf{Projektidee} \\

Die Projektidee entstammt einem spontanen Einfall.
Die beiden Mitglieder des Projektteams waren frustriert mit dem Verlauf deren früherer Projekte
und wollten etwas Neuartiges entwickeln.
Daraus entsprang die Idee, schwarze Löcher virtuell darzustellen und in Echtzeit zu simulieren.
Im Zuge der Einigung auf ein Gesamtziel versuchte das Projektteam, diese Vorstellung in einzelne
Teile zu gliedern.
In Abbildung~\ref{fig:original_notes_project_goals} sind die originalen Notizen des Projektteams zu
sehen, die bei der Überlegung der Projektziele entstanden sind.
Diese Ziele stellen die übergeordneten Gesamtausmaße des Projektes dar, welche das Team erreichen
möchte.
Im Zuge der besseren Lesbarkeit und genaueren Dokumentation werden diese sogenannten funktionalen
Anforderungen nochmals in höherer Genauigkeit aufgelistet:
\begin{itemize}
    \item   \textbf{Basis mit Dear ImGui:} Als Basis wird eine C++ Applikation aufgesetzt, welche die
            Library Dear ImGui nutzt, um dem Endnutzer die Möglichkeit zu bieten, verschiedenste Parameter
            einzustellen, als auch sich im drei dimensionalen Raum zu bewegen.
    \item   \textbf{Visualisierung:} Zum besseren Verständnis schwarzer Löcher trägt vor allem die Visualisierung
            bei.
            Deshalb sollen der Ereignishorizont, die Photonensphäre, die Akkretionsscheibe und die
            Raumzeitkrümmung dargestellt werden.
            Zudem sollen auch noch verschiedenste Parameter eines schwarzen Loches angepasst werden können.
            Das sind die Masse $m$, Position $x^i$, Geschwindigkeit $v^i$ sowie der Spin-Parameter $a$.
            Diese Punkte sollen mithilfe des Shader-Frameworks OpenGL erfüllt werden.
    \item   \textbf{Simulation:} Um über die reine Darstellung hinaus auch die unvorstellbaren Kräfte
            schwarzer Löcher zu erkennen, soll es auch die Möglichkeit geben, eine Simulation zu konfigurieren
            und danach anzusehen.
            Dabei soll es möglich sein, andere Objekte wie z.B.\ Sterne, Planeten und Raumschiffe einzufügen und
            mit dem schwarzen Loch interagieren zu lassen.
    \item   \textbf{Physikalische Korrektheit:} Damit ein Endnutzer auch lernen kann, wie schwarze Löcher in der
            realen Welt funktionieren, sollen möglichst genaue Theorien über schwarze Löcher angewendet werden.
            Zu diesen gehören z.B.\ die Kerr Metrik zur Berechnung der Raumzeitkrümmung in der Nähe rotierender
            schwarzer Löcher oder der Dopplereffekt zur Veränderung des ausgesandten Lichts.
\end{itemize} \\

\textbf{Allgemeines Ziel} \\

Weiters bildete sich über diese formulierten funktionalen Anforderungen auch ein allgemeines Ziel, das
lautet:
\begin{center}
    \textbf{Die komplizierte Physik unseres Kosmos sollte jedem Menschen so einfach und gut verständlich
    wie möglich präsentiert werden!}
\end{center}
Um den Worten des Astronomen Florian Freistetter gerecht zu werden~\autocite{Video:Podcast_Science_Communication},
soll den Menschen Wissenschaft beigebracht werden, indem man sie auf eine Reise einlädt, die Komplexität
verschiedenster Theorien weglässt und vor allem ein Vertrauen aufbaut.
Dieses Projekt soll das umsetzen und damit eine Hilfe in der Kommunikation von Wissenschaft mit der allgemeinen
Bevölkerung darstellen. \\

\textbf{These} \\

Es war nicht nötig, für dieses Projekt eine These zu formulieren, da es sich nicht um eine herkömmliche
wissenschaftliche Arbeit handelt, bei der eine Hypothese formuliert wird und im weiteren Verlauf der Arbeit
mithilfe wissenschaftlicher Methoden bestätigt oder wieder verworfen wird.
Dieses Projekt soll vielmehr ein Produkt hervorbringen, dass es den Menschen ermöglicht, auf einfache Art und
Weise Wissenschaft zu erfahren.
In diesem Projekt werden bei Erfüllung der Projektziele große Teile der Errungenschaften der Astronomie und
Kosmologie der letzten 150 Jahre umgesetzt und versucht, dem Endnutzer möglichst verständlich zu präsentieren. \\

\textbf{Informationsgewinn} \\

Die Quellen für die Recherche und den Informationsgewinn sind so zahlreich, dass sie nicht in einer Liste aufgeführt
werden können, ohne den Rahmen dieses Dokumentes zu sprengen oder die Lesbarkeit zu beeinträchtigen.
Deshalb werden hier nur einige Beispiele aufgeführt werden.
Als wichtigste Ressourcen zählen einige Projekte, die als Inspiration für die Arbeit dienen.
Sie sind in~\autocite{Project:Black_Hole_Sim_eliot, Project:Black_Hole_Sim_rossning, Project:Black_Hole_Sim_hdeu,
    Project:Black_Hole_Sim_Kamminga, Reddit:Black_Hole_Sim_Burgao} zu sehen.
Viele dieser Projekte können auf GitHub betrachtet werden, während auch auf Reddit einiges zu finden war.
Auch Bücher waren eine ordentliche Hilfestellung bei der Implementierung, wie~\autocite{Book:General_Relativity,
    Book:Black_Holes_bc} \\

\textbf{Planung des Projektablaufes} \\

Die Planung des Projektes erfolgt agil und es wurden dementsprechend keine Meilensteine vordefiniert.
Die untergeordneten Projektziele werden im Laufe des Projektes angepasst und hinzugefügt oder gelöscht.
Es wird eine Mischung der agilen Projektplanungsframeworks Scrum~\autocite{Wikipedia:Scrum} und Kanban~\autocite{
    Wikipedia:Kanban} - Scrumban~\autocite{Wikipedia:Scrumban} - ausgeführt.
Dies wird insofern eingehalten, als das Rollen für die Mitglieder des Projektteams festgelegt wurden,
wöchentliche Meetings abgehalten werden, alle drei Wochen ein Product Increment erfolgt und alles über ein
Kanban-Board verwaltet wird.
Zudem werden mittels einer Knowledge Base alle Meetings dokumentiert und andere wichtige Informationen
festgehalten.
Diese Art, ein Projekt zu planen und führen, ermöglicht eine flexible Arbeitsweise und macht besonders im Falle dieses
Projektes Sinn, da am Anfang den Mitgliedern des Projektteams nicht klar, wohin die Reise gehen soll.
Eigentlich war die Idee hinter dem Projekt eine Abwechslung in Bezug auf die restlichen Unterrichtsfächer an
der HTL Krems~\autocite{Website:HTL_Krems}.
Da alle Beteiligten des Projektes großes Engagement zeigen, wird auch außerhalb des Unterrichts gearbeitet und
versucht, die aktuellen Ziele zu erfüllen. \\

	%! Author = leo
%! Date = 1/15/26

\subsection{Projektumsetzung und Ausblick} \label{subsec:project_execution_outlook}

Dieser Abschnitt beschreibt die wesentlichen Merkmale dieses Projektes und soll die Realisierung und die aktuellen
Ergebnisse sowie das Entwicklungspotenzial erläutern. \\

\subsubsection{Innovationsgehalt und Einzigartigkeit} \label{subsubsec:innovation_uniqueness} \\

In dieser Sektion wird erklärt, warum dieses Projekt einzigartig ist und einen eigenen innovativen Charakter enthält. \\

\textbf{Innovationscharakter} \\

Dieses Projekt kann als innovativ oder außergewöhnlich angesehen werden, da schwarze Löcher für einen Großteil der
Bevölkerung ein sehr unbekanntes Thema sind und eher einschüchternd wirken, wenn man die komplexe Mathematik dahinter
betrachtet.
Außerdem geht es um besonders abstrakte Konzepte, da bis heute erst zwei echte Bilder eines schwarzen Loches angefertigt
werden konnten~\autocite{Article:EHT_Imaging_M87,Article:EHT_Imaging_SgrA}.
Demnach ist das Projekt innovativ, weil es sich mit dieser komplizierten Thematik auseinandersetzt.
Darüber hinaus wird auch versucht, anderen Menschen ein Verständnis für das Thema zu vermitteln, indem eine
interaktive Applikation entwickelt wird, die es ermöglicht, spielerisch und experimentell die komplexen
Vorgänge in der Nähe eines schwarzen Loches zu erleben. \\

\textbf{Rechtslage} \\

Das Projekt ist ein OpenSource-Projekt und der gesamte Quellcode ist auf GitHub
ersichtlich~\autocite{Project:Black_Hole_Sim_MoleHole}.
Das Projekt wurde unter der GNU General Public License v3.0 (GPL-3.0)
veröffentlicht~\autocite{License:GPL}. \\

\textbf{Einzigartigkeit} \\

Es gibt ohnehin wenig frei zugängliche Projekte, die als Ziel die Visualisierung schwarzer Löcher verfolgen.
Jedoch ist bei diesen Projekten nie eine Benutzerinteraktion vorgesehen, was der größte Unterschied zu
diesem Projekt ist.
Genau in dieser Lücke findet sich das MoleHole Projekt wieder, indem es die Echtzeitsimulation schwarzer Löcher mit einer
interaktiven Benutzeroberfläche kombiniert. \\

\subsubsection{Realisierung} \label{subsubsec:implementation} \\

Diese Sektion befasst sich mit der Implementierung der aktuellen Version der Applikation und soll
diese Schritt für Schritt beschreiben. \\

\textbf{Technologiestack} \\

Der Großteil der Applikation wurde in C++ entwickelt.
Die einzige Programmiersprache, die zusätzlich verwendet wurde, ist GLSL für die Shader-Programmierung.
Die Programmiersprachen sind also sehr einfach gehalten.
Im Gegensatz dazu werden einige Bibliotheken verwendet, um nicht alles selbst entwickeln zu müssen.
Alle verwendeten Bibliotheken sind:
\begin{itemize}
    \item   \textbf{OpenGL Libraries:} GLAD, GLFW und GLM für Parameter in GLSL und andere
            Utilities~\autocite{GitHub:Glad,GitHub:Glfw,GitHub:Glm}
    \item   \textbf{Dear ImGui:} Standardisierte Library für das \textit{User Interface}~\autocite{GitHub:ImGui}
    \item   \textbf{Spdlog:} Library für das Logging~\autocite{GitHub:Spdlog}
    \item   \textbf{PhysX:} Physiksimulationen verschiedenster Art~\autocite{GitHub:PhysX}
    \item   \textbf{Stb\_image:} Verarbeiten von Bildern~\autocite{GitHub:Stb}
    \item   \textbf{ImGuizmo:} Utility-Library für die Selektion von Objekten~\autocite{GitHub:ImGuizmo}
    \item   \textbf{Tinygltf:} Für das \textit{File-Processing} von .gltf-Dateien~\autocite{GitHub:Tinygltf}
    \item   \textbf{ImGui-Node-Editor:} Framework für einen \textit{Node Editor}~\autocite{GitHub:ImGuiNodeEditor}
    \item   \textbf{Yaml-cpp:} Verarbeitung von .yaml-Dateien~\autocite{GitHub:Yaml-cpp}
    \item   \textbf{Nativefiledialog-extended:} \textit{File-Dialogs} für den Export von Videos 
            und Bildern~\autocite{GitHub:Nativefiledialog}
\end{itemize}

Zusätzlich zu diesen Technologien, gibt es auch einige Tools und Programme, die für die Entwicklung
verwendet wurden.
Dabei wurden CLion und YouTrack von JetBrains~\autocite{Website:CLion,Website:YouTrack} und
Git~\autocite{Website:Git} als \textit{Versionskontrolle} eingesetzt. \\

\textbf{Rendering Techniken} \\

Die wichtigsten Rendering Techniken, die in der Applikation verwendet werden, sind Ray Tracing und
Ray Marching.
Beim Ray Tracing geht von der Kamera aus bis zu einem Schnittpunkt mit einem Objekt ein Lichtstrahl
aus und wird dort reflektiert und zur Lichtquelle zurückverfolgt.
Dies wird für eine große Anzahl an Schnittpunkten durchgeführt, bis schlussendlich das ganze Objekt sichtbar ist.
Dies wird dann am Bildschirm dargestellt.
Beim Ray Marching nutzt man auch Lichtstrahlen, die von der Kamera ausgehen, aber man geht hier Schritt für
Schritt und rechnet nicht sofort den Schnittpunkt mit einem Objekt aus.
Grundsätzlich ist diese Methode also rechenintensiver, aber sie ermöglicht es, gekrümmte Lichtstrahlen abzubilden.
Diese gekrümmten Lichtstrahlen sind dafür verantwortlich, dass in der Nähe eines schwarzen Loches komplexe
visuelle Effekte auftreten.
Die Lichtstrahlen werden durch die Gravitation abgelenkt und dieser sogenannte Gravitationslinseneffekt
wird durch das Ray Marching in der Applikation simuliert.
Da die Rechenleistung eines normalen Computers für das Ray Marching nicht ausreicht, muss man einen Kompromiss eingehen.
Man verzichtet auf die exakte Darstellung der Objekte und beschäftigt sich nur mit der Simulation von Lichtstrahlen,
die sehr nahe an einer großen Masse vorbeigehen.
Dadurch wird gewährleistet, dass die gewünschten optischen Effekte zu sehen sind, jedoch auch nicht zu viel Genauigkeit
verloren geht, da die Lichtstrahlen in größerer Entfernung nicht mehr wirklich sichtbar abgelenkt werden.
Für die Software werden also beide Rendering Techniken eingesetzt: Ray Tracing um Lichtstrahlen bis zu einer gewissen
Entfernung zu schwarzen Löchern zu simulieren und Ray Marching um die Lichtstrahlen in der Nähe von schwarzen
Löchern zu berechnen. \\

%Antialiasing (FXAA);
%Bloom Effekt;
%Third-Person-Kamera;
%irst-Person-Kamera; \\

\textbf{Physikalische Effekte} \\

Die physikalischen Effekte, die bis zum jetzigen Stand in der Software realisiert worden sind,
bilden die Grundlage für die Simulation schwarzer Löcher.

Der wichtigste Effekt betrifft den Gravitationslinseneffekt, der sich durch die starke
Raumzeitkrümmung in der Nähe eines schwarzen Loches ergibt.
Hierbei werden Lichtstrahlen in der Nähe einer großen Masse stärker abgelenkt als in größerer Entfernung.
Dies entspricht dann einem ähnlichen Effekt wie bei einer herkömmlichen \textit{Konvexlinse}.
Um diesen Effekt visuell darzustellen, ist es nötig die \textit{Nullgeodätische} auszurechnen, welche den
Weg des Lichts um ein schwarzes Loch angibt~\autocite{Book:General_Relativity}.
Man nehme an, dass $x(\lambda)$ den Weg eines Objektes durch Raum und Zeit und $\Gamma$ das
\textit{Christoffel-Symbol} angibt.
Damit ergibt sich nach Lösung von
\begin{equation}
    \frac{d^2 x^{\rho}}{d\lambda^2} + \Gamma^{\rho}_{\mu \nu} \frac{d x^{\mu}}{d\lambda} \frac{d x^{\nu}}{d\lambda} = 0
    \label{eq:geodesic_equation}
\end{equation}
genau dieser Weg~\autocite{Book:Gravitation,Book:General_Relativity}.
Für diese Lösung gibt es in der \textit{Schwarzschild-Metrik} eine vereinfachte Form, die in der Applikation
tatsächlich verwendet wird.
Die Kraft $\vec{F}(r)$ ergibt sich aus
\begin{equation}
    \vec{F}(r) = \frac{3h^2\hat{r}}{2r^5}
    \label{eq:schwarzschild_acceleration}
\end{equation}
mit einer Konstante $h$, dem Radius $r$ und der Basis $\hat{r}$~\autocite{Website:Starless,Website:Eliot_Han}.

Die Theorie, die vorerst zur Simulation der Gravitation verwendet wird, ist die \textit{Gravitationstheorie}
von Isaac Newton.
Diese Theorie ist schon sehr alt, funktioniert aber immer gut für die Berechnung der Anziehungskräfte
zwischen Objekten mit geringen Massen.
$G$ sei die Gravitationskonstante, $M$ und $m$ die Massen zweier Objekte und $\vec{r}$ der Abstand
zwischen den Zentren der beiden Objekte.
Die Gravitationskraft $\vec{F}_G$ berechnet sich dann durch~\autocite{Book:Principia_Mathematica}
\begin{equation}
    \vec{F}_G = \frac{GMm}{\vec{r}^2}.
    \label{eq:gravitational_force}
\end{equation}

Für die Visualisierung der Raumzeitkrümmung wird die Schwarzschild-Metrik verwendet.
Diese wird zur Berechnung der Krümmung in der Nähe einer Punktmasse verwendet.
Es seien die Distanz zwischen zwei Ereignissen $ds$, der Schwarzschildradius $R_S$, die Entfernung
vom Zentrum der Masse $R$ sowie die Dimensionen der Raumzeit $dt$, $dr$ und $d\Omega$ gegeben.
Dann beschreibt
\begin{equation}
    ds^2 = -\Bigg(1 - \frac{R_S}{R}\Bigg)dt^2 + \frac{1}{\Big(1 - \frac{R_S}{R}\Big)}dr^2 + r^2 d\Omega^2
    \label{eq:schwarzschild_metric}
\end{equation}
die Raumzeitkrümmung in der Nähe einer
Punktmasse~\autocite{Book:Gravitation,Book:General_Relativity,Book:Black_Holes_bc}.

Ein weiterer wichtiger Effekt ist die richtige Temperaturskalierung in der Akkretionsscheibe und die damit verbundene
Lichtemission, die vom menschlichen Auge oder einem Bildsensor als Farbe erkannt wird.
Man nehme an, dass $G$ die Gravitationskonstante, $\sigma$ die Stefan-Boltzmann-Konstante,
$M$ die Masse des schwarzen Loches, $\dot{M}$ die Akkretionsrate, $R_i$ der innere Radius der Akkretionsscheibe
und $R$ der aktuelle Radius in der Akkretionsscheibe ist.
Die Temperaturverteilung $T(R)$ in der Akkretionsscheibe lässt sich dann durch
\begin{equation}
    T(R) = \Bigg[ \frac{3GM\dot{M}}{8\pi \sigma R^3} \Bigg( 1 - \sqrt{\frac{R_i}{R}}\Bigg) \Bigg]^{\frac{1}{4}}\label
    {eq:temp_scaling_accretion_disk}
\end{equation}
beschreiben~\autocite{Article:Temp_Accretion_Disk}. \\

%Physikalische Effekte:

%Dopplereffekt;
%Gravitative Rotverschiebung;
%Akkretionsscheibe;
%ISCO;
%Photonensphäre;
%Schwarzkörperstrahlung;
%Hertzsprung-Russell-Diagramm; \\

\textbf{Forschungsmethoden} \\

Die verwendete Methodik, um Erkenntnisse zu gewinnen, ist die Empirie.
Sie dient der Verbesserung der Applikation.
Es wurde ein Problem identifiziert und danach an einer vermeintlichen Lösung gearbeitet.
Nachdem man eine Lösung auf dem Papier entwickelt hatte, wurde diese in der Applikation implementiert und ausprobiert.
Durch die Überprüfung der neuen Implementierung konnte eine Entscheidung getroffen werden, ob die Lösung
zielführend war oder nicht.
Genau so musste das Projektteam die Performance verbessern.
Es wurde festgestellt, das die Software die gewünschte Leistung nicht lieferte, also musste sich etwas überlegt werden.
Deswegen wurde mithilfe einiger Zeichnungen an der Tafel eine mögliche Lösung erarbeitet.
Diese bezieht sich auf die oben genannte Kombination aus Ray Tracing und Ray Marching.
Nach der Implementierung dieser Lösung konnte festgestellt werden, dass die Performance
signifikant verbessert wurde.

Ein weiteres Mittel für die Gewinnung wichtiger Erkenntnisse ist die Arbeitsweise mit Sprints wie
in~\ref{subsubsec:planning} erklärt.
Sie ermöglicht die Reflexion über die vergangene Arbeit und eine kontinuierliche Verbesserung
des Projektes.
Außerdem sind in diesem Format regelmäßige Erweiterungen der Software geplant - sogenannte
\textit{Product Increments} - die vom agilen Projektmanagement vorgegeben sind und andauernd für
eine verbesserte Version des Produktes sorgen.

Weiters war die Recherche ein wichtiges Mittel, um an nötige Informationen zu gelangen.
Hierfür wurden wissenschaftliche Artikel, Bücher und Videos wie zum
Beispiel~\autocite{Book:General_Relativity,Book:Black_Holes_bc,Video:Coding_Adventure_Clouds}
verwendet. \\

\subsubsection{Ergebnisse} \label{subsubsec:results} \\

Diese Sektion beschäftigt sich mit dem Fortschritt, der bis heute erreicht werden konnte.
Dabei werden einige Bilder zu sehen sein und anhand einiger Zahlen der Umfang beschrieben. \\

\FloatBarrier
\begin{figure}[!bth]
    \centering
    \includegraphics[width=0.9\textwidth]{img/Export-JI-10}
    \caption{Beispiel des User-Interfaces der Applikation.}
    \label{fig:user_interface}
\end{figure}

\textbf{Ergebnisse in visueller Darstellung} \\

Die folgende Liste wird die Bereiche des User-Interfaces (UIs) der Applikation erklären und
minimal auf die Funktionsweise der Unterpunkte eingehen.
In Abbildung~\ref{fig:user_interface} ist ein Beispiel des UIs abgebildet.
Die nummerierten Bereiche werden nachstehend erläutert:
\begin{itemize}
    \item   \textbf{1 - Objekte:} In diesem Bereich des Menüs ist es möglich, Objekte hinzuzufügen, zu löschen
            und zu bearbeiten.
            Es können drei verschiedene Arten an Objekten verwendet werden: schwarze Löcher, Sterne~\&~Planeten
            sowie verschiedene 3D-Modelle aus .gltf-Dateien.
            In dieser Szene wurde beispielhaft ein schwarzes Loch hinzugefügt.
            Es ist zu erkennen, dass für ein schwarzes Loch die Parameter Masse $m$, Spin $a$, Position $x^i$ sowie
            die Drehachse $a^i$ anpassbar sind.
    \item   \textbf{2 - Visualisierungen:} Dieses Fenster ermöglicht die Anpassung verschiedenster Effekt, die im
            Prozess des Renderings verwendet werden.
            Dazu zählen die Aktivierung des Gravitationslinseneffektes, der gravitativen Rotverschiebung, des
            Volumetric Cloud Renderings sowie des Dopplereffektes.
            Weiters lassen sich die Höhe der Akkretionsscheibe, die Skalierung der \textit{Noise-Funktion}, die für
            die Darstellung der Akkretionsscheibe genutzt wird, die Details in der Scheibe, als auch die Geschwindigkeit
            $\omega$ mit der sich die Scheibe dreht, einstellen.
            Weiters gibt es die Möglichkeit, den \textit{Bloom-Effekt} zu aktivieren und einige Einstellungen dafür, die
            man verändern kann.
            Zusätzlich ist es auch möglich, das \textit{Anti-Aliasing} einzuschalten, welches für eine verbesserte
            Qualität des Bildes der Szene sorgt.
            Die Einstellungen, die in diesem Menü aber die größten Möglichkeiten bieten, sind die verschiedenen
            \textit{Debug-Layer}, die angezeigt werden können.
            Es gibt hier die Auswahl zwischen den \textit{Object Paths}, welche die Pfade der Objekte während einer
            Simulation anzeigen und dem \textit{Gravity Grid}, welches die Raumzeitkrümmung in einem dreidimensionalen
            Gitternetz darstellt.
            Diese Debug-Layer sind in Abbildung~\ref{fig:debug_layers} zu sehen.
    \item   \textbf{3 - Kamera:}
            Der Bereich für die Konfiguration der Kamera ist hier zu finden.
            Die meisten Einstellungen in diesem Menüpunkt werden meistens nicht gebraucht, da
            sich auch mit den Tasten W, A, S, D, Q und E fortbewegt werden kann.
            Allerdings können Fortbewegungsgeschwindigkeit und das Sichtfeld angepasst werden.
            Zudem sind hier auch die Einstellungen für die Perspektiven zu finden.
            Wenn die \textit{Third-Person-Perspektive} aktiviert wird, ist es möglich ein 3D-Modell als
            Kamera-Objekt auszuwählen und dieses durch eine Szene zu bewegen.
            In Abbildung~\ref{fig:third_person_perspective} kann dies angesehen werden.
    \item   \textbf{4 - Simulation:} Diese beiden Icons bieten die Möglichkeit, eine Simulation
            zu starten, zu pausieren oder zu stoppen.
            Wenn die Simulation gestartet wird, werden auch alle Objekte in den Ursprungszustand zurückgesetzt.
    \item   \textbf{5 - Seitenleiste:} Die Seitenleiste soll dem Benutzer eine bessere Übersicht geben und
            Funktion bereitstellen, die Menüs ein- und auszublenden.
            Weiters sind hier auch die Einstellungen untergebracht, die für das Setzen eines standardmäßigen
            Export-Pfades und das Anpassen der Schriftgröße und Schriftart eine Möglichkeit bieten.
    \item   \textbf{6 - Szene:} Dies ist der Bereich, in dem die hauptsächliche Simulation und Visualisierung
            eingebettet ist und somit das Kernstück der Software ist.
            Hier kann man alle Einstellungen als finales Bild ansehen und eine Simulation beobachten.
\end{itemize}

\FloatBarrier
\begin{figure}[!bth]
    \centering
    \begin{subfigure}{0.45\textwidth}
        \centering
        \includegraphics[width=\linewidth]{img/Export-JI-12}
        \caption{Rendering eines schwarzen Loches mit weißer Akkretionsscheibe}
        \label{fig:black_hole_rendering_white}
    \end{subfigure}
    \begin{subfigure}{0.45\textwidth}
        \centering
        \includegraphics[width=\linewidth]{img/Export-JI-13}
        \caption{Rendering eines schwarzen Loches mit Akkretionsscheibe und realistischer Lichtemission.}
        \label{fig:black_hole_rendering_black}
    \end{subfigure}
    \caption{Rendering schwarzer Löcher mit Akkretionsscheibe und Gravitationslinseneffekt.}
    \label{fig:black_hole_rendering}
\end{figure}

Die beiden Rendering-Ergebnisse in Abbildung~\ref{fig:black_hole_rendering} zeigen jeweils ein
einziges schwarzes Loch.
Bei dem linken Bild in Abbildung~\ref{fig:black_hole_rendering_white} ist die Akkretionsscheibe als weiß zu
erkennen.
Das kommt daher, da dieses Rendering in einem früheren Stadium des Projektes aufgenommen worden ist und
die Temperaturskalierung nicht den realen Verhältnissen entsprochen hat.
Im rechten Bild in Abbildung~\ref{fig:black_hole_rendering_black} kann man eine genauere Verteilung der
Temperatur in der Akkretionsscheibe erkennen, welche sich in Abhängigkeit der Entfernung zum Zentrum
des schwarzen Loches ändert.
Diese Temperaturverteilung wird in der Gleichung~\ref{eq:temp_scaling_accretion_disk}
mathematisch beschrieben.

\FloatBarrier
\begin{figure}[!bth]
    \centering
    \begin{subfigure}{0.45\textwidth}
        \centering
        \includegraphics[width=\linewidth]{img/Export-JI-03}
        \caption{First-Person-Perspektive}
        \label{fig:first_person_perspective}
    \end{subfigure}
    \begin{subfigure}{0.45\textwidth}
        \centering
        \includegraphics[width=\linewidth]{img/Export-JI-04}
        \caption{Third-Person-Perspektive}
        \label{fig:third_person_perspective}
    \end{subfigure}
    \caption{Vergleich der First-Person- und Third-Person-Perspektive.}
    \label{fig:camera_perspectives}
\end{figure}

In der Software wurden zwei verschiedene Kameraperspektiven umgesetzt: die First-Person-Perspektive und
die Third-Person-Perspektive.
In Abbildung~\ref{fig:camera_perspectives} kann man die beiden im Vergleich betrachten.
Für die Third-Person-Perspektive kann man ein Objekt auswählen, das man dann steuern kann.
Dieses Objekt wird mit einem zusätzlichen Würfel, welcher eine Art Crosshair darstellen soll, in der
Applikation gerendert.

\FloatBarrier
\begin{figure}[!bth]
    \centering
    \begin{subfigure}{0.45\textwidth}
        \centering
        \includegraphics[width=\linewidth]{img/Export-JI-05}
        \caption{Object Paths Debug-Layer}
        \label{fig:debug_layer_object_paths}
    \end{subfigure}
    \begin{subfigure}{0.45\textwidth}
        \centering
        \includegraphics[width=\linewidth]{img/Export-JI-06}
        \caption{Gravity Grid Debug-Layer}
        \label{fig:debug_layer_gravity_grid}
    \end{subfigure}
    \caption{Beispiele der Debug-Layer zur Visualisierung komplexer Phänomene.}
    \label{fig:debug_layers}
\end{figure} \\

Weiters ist auch das Anregen der Vorstellungskraft ein wichtiger Punkt.
Hierfür gibt es die Debug-Layer, welche in Abbildung~\ref{fig:debug_layers} zu sehen sind.
Diese Visualisierungen sollen dem Benutzer helfen, sich die physikalischen Effekte besser
vorstellen zu können.
Da es leider nicht möglich ist, sich eine 4-dimensionale Raumzeit vorzustellen, muss dem Gehirn
ein wenig auf die Sprünge geholfen werden.
Diese Tricks nutzt die Applikation, um ein besseres Verständnis für die komplexen und abstrakten
Konzepte der Physik zu vermitteln. \\

%Dreifachsternsystem:
%Performance:

\FloatBarrier
\textbf{Erreichung der Ziele} \\

Das Projektteam setzt sich alle Ziele am Anfang eines sogenannten Sprints~\autocite{Website:Scrum} und schreibt
diese in der Projektmanagement-Software YouTrack~\autocite{Website:YouTrack} nieder.
Dort werden allgemein die Ziele formuliert, mit einer Priorität versehen, der Arbeitsaufwand anhand von Story Points
geschätzt und ein Verantwortlicher festgelegt.
Um zu überprüfen, ob das Projektteam die Ziele erfüllt, werden diese in der Applikation von den Mitgliedern des
Projektteams ausprobiert.
Da dies keine besonders zuverlässige Methode zur Bestimmung der erfüllten Ziele ist, gibt es am Ende eines jeden
Sprints ein Sprint-Review Meeting, bei dem das Team ihre Ergebnisse dem Projektbetreuer sowie einem ausgewählten Kunden
(ein Mitschüler) vorstellt und den Sprint somit abschließt~\autocite{Website:Scrum}.
Danach werden erneut Ziele für die nächste Iteration überlegt. \\

\subsubsection{Entwicklungspotenzial} \label{subsubsec:potential} \\

Es steckt noch einiges an Potenzial in diesem Projekt und das Team plant eine regelmäßige Erweiterung
der Software, um diesem Potenzial gerecht zu werden.
Demnach werden in dieser Sektion alle potenziellen Erweiterungen des Projektes aufgezählt. \\

\textbf{Gesellschaftliche Auswirkungen} \\

Wie bereits in~\ref{subsec:project_origin_and_planning} erwähnt ist das Gesamtziel des Projektes eine verbesserte
Wissenschaftskommunikation durch eine Applikation mit der man experimentell und spielerisch komplexe Vorgänge
unseres Universums erleben und erlernen kann.
Das Produkt ist so konzipiert, dass es die Möglichkeit gibt, eigene Simulationen zu erstellen und in diesen zu experimentieren,
aber auch über Tooltips und Dokumentation Genaueres über die simulierten physikalischen Effekte zu erfahren.
Weiters ist die Applikation als OpenSource-Software frei auf GitHub verfügbar und wird dies in Zukunft auch bleiben, was
den gesellschaftlichen Nutzen erheblich verbessert, weil es jeder, der einen PC besitzt, ausprobieren kann.
Insgesamt lässt sich feststellen, dass es dem Projektteam ein Anliegen ist, mit dieser Applikation die
Wissenschaftskommunikation zu verbessern und damit mehr Menschen den Zugang zur Wissenschaft zu erleichtern. \\

\textbf{Kooperationen} \\

Aktuell gibt es bei diesem Projekt keine bestehenden Kooperationen, jedoch würde das Projektteam in Zukunft welche anstreben.
Vor allem in der Zukunft möchte das Projektteam die Applikation weiter verbessern, indem im zukünftigen Studium
wichtige Erkenntnisse gesammelt werden und mit professionellen Wissenschaftlern zusammengearbeitet wird.
Zusätzlich wird mit Kooperationen auch die Entwicklung weiterer Forschungsarbeiten angestrebt. \\

\textbf{Releases} \\

Wenn die Entwicklung des Projektes weiterhin große Fortschritte erzielt, ist eine Veröffentlichung des Projektes
zum Beispiel auf Steam anzuvisieren.
Aktuell kann das Projekt zwar schon benutzt werden, da es ein öffentliches GitHub Repository mit dem Code gibt, aber
es ist ein wenig technisches Wissen notwendig, um die Applikation zu verwenden.
Wenn die Applikation auf einer Vertriebsplattform wie Steam veröffentlicht werden kann, würde das den Zugang
zu der Software erleichtern.
Insgesamt kann die Aussage getroffen werden, dass das Projekt in Zukunft offiziell veröffentlicht werden soll, aber
noch viel Arbeit notwendig ist, um dieses Ziel zu erreichen.
Hierfür muss vor allem noch eine umfangreiche Dokumentation angefertigt werden und die technischen Potenzial sollten
möglichst gut ausgeschöpft sein.
Hinzu kommt die Entfernung aller Fehler und Unannehmlichkeiten aus der Software. \\

\textbf{Technische Potenziale} \\

Das mit Abstand wichtigste physikalische Phänomen, das in der Applikation noch umgesetzt werden muss, ist die
Kerr-Metrik, welche im Jahre 1963 vom neuseeländischen Physiker Roy Kerr als Lösung der Feldgleichungen aus der
ART aufgestellt worden ist~\autocite{Book:General_Relativity}.
Hierfür wird allerdings Wissen aus der ART benötigt, das sich das Projektteam noch nicht angeeignet hat.
Deshalb wird es noch einige Zeit dauern, bis dieses Feature verwirklicht werden kann.
Nichtsdestotrotz beschäftigt sich das Team mit diesem Effekt, da dieser extrem relevant für eine akkurate Darstellung
eines schwarzen Loches ist.
Mit der Kerr-Metrik folgen einige andere wichtige Phänomene aus der ART, die als mögliche Ziele des Projektes
im Hinterkopf behalten werden sollen, jedoch erst nach Umsetzung der Kerr-Metrik implementiert werden können.
Diese Effekte sind das Frame-Dragging, die Zeitdilatation, die Längenkontraktion sowie die Gravitation im Allgemeinen.
Vor allem Zeitdilatation und Längenkontraktion sind wichtig für eine genaue Simulation schwarzer Löcher, da für Beobachter
außerhalb der unmittelbaren Umgebung eines schwarzen Loches Objekte nahe dem Ereignishorizont sich immer langsamer
bewegen und letztendlich verblassen~\autocite{Book:Gravitation}.
Hierbei muss natürlich auch die Gravitation implementiert werden, die der wichtigste Bestandteil der zukünftigen Ziele
sein wird.
Es soll dabei die aktuell verwendete Simulation der Gravitation nach der Gravitationstheorie von Isaac
Newton~\autocite{Book:Principia_Mathematica} durch die Gravitationstheorie von Albert Einstein, die ART, ersetzt werden.
Mit dieser Theorie wäre es möglich, Objekte, die sich sehr nahe an großen Massen befinden, genauestens zu simulieren.
Generell lässt sich sagen, dass die Theorie der Gravitation von Isaac Newton in der Nähe von schwarzen Löchern nicht
funktionieren wird, da die Raumzeit hier besonders stark gekrümmt ist~\autocite{Book:Gravitation,Book:General_Relativity}.
Aus der Gravitationstheorie lässt sich ein weiterer Effekt ableiten, der für eine genaue Simulation von Objekten in der
Nähe großer Massen von Relevanz ist: die Gezeitenkraft~\autocite{Paper:Tidal_forces}.
Mit der Implementierung dieser Kraft wäre es möglich, Himmelskörper wie zum Beispiel Millers Planet aus dem Film
Interstellar~\autocite{Film:Interstellar,Book:Science_of_Interstellar} physikalisch möglichst genau zu simulieren.
Aus technischer Perspektive soll die visuelle Darstellung der Objekte auch noch verbessert werden.
Hierfür wird das \textit{Volumetric Cloud Rendering} verwendet werden, um die Akkretionsscheibe eines schwarzen Loches
detailreicher darzustellen.
Diese Rendering-Technik kümmert sich um die Belichtung wolkenartiger Gebilde, da Licht in Wolken gestreut und abgelenkt wird,
sodass nicht genau das Licht den Beobachter erreicht, das ursprünglich in seine Richtung ausgesandt worden
ist~\autocite{Video:Coding_Adventure_Clouds,Paper:Volumetric_Cloud_Rendering}.
Außerdem wird eine Partikel Simulation angestrebt, sodass man sehen kann, wie sich über die Zeit eine Akkretionsscheibe um
ein schwarzes Loch bildet.
Abgesehen von der technischen Simulation, ist es auch das Ziel, Wissenschaft zu kommunizieren.
Daher ist auch geplant, Erklärungen und Beschreibungen mittels Tooltips und detaillierte Berichte in einem Menü unterzubringen.
Diese Erklärungen sollen auch nicht auf Mathematik verzichten, sollen diese aber verständlich erklären und wenn nötig auf externe
Ressourcen verweisen.
Somit wäre garantiert, dass interessierte Menschen relativ einfach an die nötigen Ressourcen gelangen können, die notwendig sind, um
komplexe physikalische Konzepte zu verstehen. \\

\textbf{Wirtschaftliches Potenzial} \\

Der \textit{Entrepreneurship-Charakter} besteht aus der Möglichkeit, durch das Projekt
Erfahrungen und Erkenntnisse zu sammeln und diese in zukünftigen Projekten anzuwenden.
Der wirtschaftliche Nutzen dieses Projektes besteht also indirekt, indem dieses Projekt als
Chance angesehen wird, um Wissen zu erlangen und dieses in zukünftigen Projekten wirtschaftlich
erfolgreich anzuwenden.
Deshalb wird auch genauestens auf ein sorgfältiges Projektmanagement geachtet, um diese Erfahrung
später professionell nutzen zu können.
Es ist dem Projektteam wichtig, Wissenschaft für alle kostenlos zugänglich zu machen, weshalb dieses
Projekt rein dem Erkenntnisgewinn und der Wissenschaftskommunikation dient. \\

    \textbf{Zusammenarbeit im Team:}

Die Zusammenarbeit im Team verlief durchwegs positiv.
Beide Mitglieder brachten ihre individuellen Stärken ein und ergänzten sich dadurch ausgezeichnet.
Es wurden viele Probleme gemeinsam besprochen und gelöst.
Auch die Zusammenarbeit mit dem Projektbetreuer war stets konstruktiv und hilfreich.
Er gibt dem Entwicklungsteam regelmäßig Feedback, welches zur Verbesserung des Projektes beiträgt.

\textbf{Kompetenzen im Team:}

Gabriel hat bereits Erfahrung in der Entwicklung von Applikationen im Bereich Computergrafik mit C++ und OpenGL\@.
Dies war besonders hilfreich zu Beginn des Projektes, da er das Grundgerüst der Applikation schnell aufsetzen konnte.
Er ist auch in den Bereichen Software Architektur und Performanceoptimierung sehr kompetent.

Leonhard bringt viel Wissen im Bereich der theoretischen Physik mit,
welches für die korrekte Umsetzung der physikalischen Effekte essenziell ist.
Zudem ist auch er ein talentierter Programmierer mit der Fähigkeit, sich schnell in neue Technologien einzuarbeiten.

Beide Mitglieder verfügen über viel Allgemeinwissen im Bereich moderner Softwareentwicklungsmethoden,
was die Organisation und Planung des Projektes erleichtert hat.

Viele der Grundkenntnisse der Physik und Informatik wurden im Rahmen der Ausbildung an der HTL Krems erworben.
Vertiefende Kenntnisse in den Bereichen Computergrafik,
Softwareentwicklung und theoretische Physik wurden durch eigenständige Recherche
und das Studium von Fachliteratur erlangt.

\textbf{Aufgabenverteilung:}

Die Aufgabenverteilung im Team erfolgte größtenteils nach den individuellen Stärken und Interessen der Mitglieder.
Gabriel übernahm hauptsächlich die Implementierung des Applikationsgerüsts,
die Integration von OpenGL und die Optimierung der Rendering-Performance.
Leonhard konzentrierte sich auf die Implementierung der physikalischen Modelle,
die Berechnung der Raumzeitkrümmung und die Simulation der Interaktionen zwischen Objekten.
Beide Mitglieder arbeiten jedoch eng zusammen und unterstützten sich gegenseitig bei der Lösung von Problemen.

In vielen Bereichen wie UI Design, Rendering und Research arbeiten beide Mitglieder gemeinsam,
um die bestmöglichen Ergebnisse zu erzielen.

\textbf{Schulische Projektbetreuung:}

Der Projektbetreuer, Jürgen Katzenschlager,
hatte zwar keine tiefgehenden Kenntnisse im Bereich der Computergrafik oder theoretischen Physik,
konnte aber durch seine Erfahrung in der Projektleitung und Softwareentwicklung wertvolle Ratschläge geben.
Er half dem Team,
den Fokus auf die wichtigsten Ziele zu legen, und unterstützte bei der Planung und Organisation des Projektes.
Seine regelmäßigen Reviews und Feedback-Sessions trugen dazu bei,
dass das Projekt auf Kurs blieb und die Qualität der Arbeit hoch war.

\textbf{Koordination:}

Die Umsetzung des Projektes erfolgt auf der Basis agiler Methoden mit 3 Wochen Sprints.
Wöchentliche Status-Meetings werden abgehalten, um den Fortschritt zu besprechen und eventuelle Probleme zu lösen.
Am Ende jedes Sprints findet ein Sprint-Review statt, bei dem die erreichten Ziele präsentiert und bewertet werden.
Die Aufgaben werden in einem Kanban-Board in YouTrack verwaltet,
wo auch der Fortschritt und die Prioritäten der einzelnen Aufgaben festgehalten werden.

Die Entwicklung verlief bisher größtenteils planmäßig,
obwohl es natürlich auch einige Herausforderungen gab,
besonders im Rendering und Realtime Performance Bereich,
welche zu leichten Verzögerungen führten.
An wichtigen Entscheidungspunkten fand sich das Team in einem Brainstorming-Meeting zusammen,
um die bestmögliche Lösung zu erarbeiten, welche anschließend erfolgreich umgesetzt wurde.

Die offene Kommunikation, das gegenseitige Verständnis und die gegenseitige Unterstützung
haben dazu beigetragen, dass das Team harmonisch zusammenarbeitet.
Bisher sind keine Konflikte im Team aufgetreten.

\textbf{Zeitaufwand und Kosten:}

Zum Stand des 19.1.2026 betrug der gesamte Zeitaufwand für das Projekt etwa 250 Stunden.
Dies umfasst sowohl die Entwicklungszeit als auch die Zeit für Recherche, Planung und Meetings.
Die Kosten für das Projekt waren minimal, da alle benötigten Werkzeuge und Ressourcen kostenlos verfügbar waren.




	% #########################################################################################
	%
	%		Section Einsatz von KI
	%
	% #########################################################################################

	\section{Einsatz von KI} \label{sec:use_of_ai}

	%! Author = leo
%! Date = 1/22/26

Dieser Abschnitt soll einen Überblick darüber geben, wie
dieses Projekt mit der Technologie der künstlichen Intelligenz (KI)
umgegangen ist und inwiefern diese eine Rolle im Projektverlauf gespielt hat.

\subsection{Einsatz in der Programmierung} \label{subsec:implementation_ai}

In diesem Projekt wird KI zur Steigerung der Effizienz und zur Erlernung verschiedener
Techniken genutzt.
Dabei werden hauptsächlich die Claude Sonnet Modelle von Anthropic verwendet, die
sich als äußerst nützlich für die Entwicklung erwiesen
haben~\autocite{Article:Sonnet_4.5,Article:Sonnet_4.0,Article:Sonnet_Chess}.
Allgemein hat sich der Einsatz von KI-Tools gelohnt und das Projekt vorangetrieben.
Vor allem beim Erlernen der Shaderprogrammierung war die KI hilfreich, da sie Code geschrieben
hat, der ausprobiert und erlernt werden konnte.
Jedoch haben sich auch Grenzen beim Verwenden dieser Tools gezeigt.
Diese bestehen vor allem in vielen Fehlern, die von der KI gemacht werden und
danach händisch ausgebessert werden müssen.
Auch die Codequalität erfüllt nicht immer die Standards, die sie erfüllen sollte.
Aufgrund dieser Erfahrungen mit der neuen Technologie, wird viel
Wert auf das Erlernen der Funktionsweise bestimmter Rendering Techniken und
physikalischer Prozesse gelegt, damit die Qualität des Codes den Erwartungen
des Projektteams gerecht werden kann.

\subsection{Einsatz in der Dokumentation}\label{subsec:documentation_ai}

Im Bereich der Projektdokumentation bietet sich ein anderes Bild.
Hier wurde bewusst auf KI verzichtet, da die Qualität des geschriebenen Inhaltes
die höchste Priorität hatte.
Außerdem ist der Einsatz von KI-Tools auch bezüglich des Urheberrechts eine
gewisse Grauzone.
Deshalb könnten möglicherweise Probleme auftreten, mit
denen das Team nichts zu tun haben möchte.



	% #########################################################################################
	%
	%		Section Bibliography
	%
	% #########################################################################################

	\newpage
	\section{Literaturverzeichnis} \label{sec:literaturverzeichnis}
	\printbibliography[heading=none, title={Literaturverzeichnis}]

	\newpage
	\section{Abbildungsverzeichnis} \label{sec:abbildungsverzeichnis}
	%! suppress = Makeatletter
	\makeatletter
	\renewcommand\listoffigures{%
		\@starttoc{lof}%
	}
	\makeatother
	\listoffigures

\end{document}