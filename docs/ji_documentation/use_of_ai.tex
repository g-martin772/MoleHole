%! Author = leo
%! Date = 1/22/26

Dieser Abschnitt soll einen Überblick darüber geben, wie
dieses Projekt mit der Technologie der künstlichen Intelligenz (KI)
umgegangen ist und inwiefern diese eine Rolle im Projektverlauf gespielt hat.

\subsection{Einsatz in der Programmierung} \label{subsec:implementation_ai}

In diesem Projekt wird KI zur Steigerung der Effizienz und zur Erlernung verschiedener
Techniken genutzt.
Dabei werden hauptsächlich die Claude Sonnet Modelle von Anthropic verwendet, die
sich als äußerst nützlich für die Entwicklung erwiesen
haben~\autocite{Article:Sonnet_4.5,Article:Sonnet_4.0,Article:Sonnet_Chess}.
Allgemein hat sich der Einsatz von KI-Tools gelohnt und das Projekt vorangetrieben.
Vor allem beim Erlernen der Shaderprogrammierung war die KI hilfreich, da sie Code geschrieben
hat, der ausprobiert und erlernt werden konnte.
Jedoch haben sich auch Grenzen beim Verwenden dieser Tools gezeigt.
Diese bestehen vor allem in vielen Fehlern, die von der KI gemacht werden und
danach händisch ausgebessert werden müssen.
Auch die Codequalität erfüllt nicht immer die Standards, die sie erfüllen sollte.
Aufgrund dieser Erfahrungen mit der neuen Technologie, wird viel
Wert auf das Erlernen der Funktionsweise bestimmter Rendering Techniken und
physikalischer Prozesse gelegt, damit die Qualität des Codes den Erwartungen
des Projektteams gerecht werden kann.

\subsection{Einsatz in der Dokumentation}\label{subsec:documentation_ai}

Im Bereich der Projektdokumentation bietet sich ein anderes Bild.
Hier wurde bewusst auf KI verzichtet, da die Qualität des geschriebenen Inhaltes
die höchste Priorität hatte.
Außerdem ist der Einsatz von KI-Tools auch bezüglich des Urheberrechts eine
gewisse Grauzone.
Deshalb könnten möglicherweise Probleme auftreten, mit
denen das Team nichts zu tun haben möchte.
