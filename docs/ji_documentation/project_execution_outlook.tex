%! Author = leo
%! Date = 1/15/26

\subsection{Projektumsetzung und Ausblick} \label{subsec:project_execution_outlook}

\textbf{Innovationscharakter}



\textbf{Rechtslage}



\textbf{Einzigartigkeit}

Es gibt ohnehin wenige Projekte, die als Ziel die Visualisierung schwarzer Löcher verfolgen.
Man könnte diese an einer Hand abzählen.
Jedoch ist bei diesen Projekten nie eine Benutzerinteraktion vorgesehen, was die Komplexität der technischen Umsetzung
nicht schmälert, aber auch keine Endanwendung finden wird.
Genau in dieser Lücke findet sich das MoleHole Projekt wieder, indem es die Echtzeitsimulation schwarzer Löcher mit einer
interaktiven Benutzeroberfläche kombiniert.

\textbf{Realisierung}

Technologien:

ImGui, OpenGL


Rendering Techniken:

Ray Tracing;
Ray Marching;
Kombination aus Ray Marching und Ray Tracing;
Antialiasing (FXAA);
Bloom Effekt;
Third-Person-Kamera;
First-Person-Kamera;


Physikalische Effekte:

Gravitation (Newton);
Gravitational Lensing;
Dopplereffekt;
Gravitative Rotverschiebung;
Akkretionsscheibe;
Temperaturverteilung der Akkretionsscheibe;
Schwarzschild-Metrik;
ISCO;
Photonensphäre;
Schwarzkörperstrahlung;
Hertzsprung-Russell-Diagramm;


Geplant:

Kerr-Metrik;
Frame Dragging;
Zeitdilatation;
Längenkontraktion;
Gravitation (Einstein);
Volumetric Cloud Rendering;
Partikel-Simulation;
Gezeitenkräfte;



\textbf{Forschungsmethoden}



\textbf{Aktuelle Ergebnisse}



\textbf{Erreichung der Ziele}



\textbf{Gesellschaftliche Auswirkungen}



\textbf{Kooperationen}



\textbf{Entwicklungspotenzial}



\textbf{Wirtschaftliches Potenzial}