%! Author = leo
%! Date = 1/15/26

\subsection{Projektumsetzung und Ausblick} \label{subsec:project_execution_outlook}

Dieser Abschnitt beschreibt die wesentlichen Merkmale dieses Projektes und soll die Realisierung und die aktuellen
Ergebnisse sowie das Entwicklungspotenzial erläutern. \\

\subsubsection{Innovationsgehalt und Einzigartigkeit} \label{subsubsec:innovation_uniqueness} \\

In dieser Sektion wird erklärt, warum dieses Projekt einzigartig ist und einen eigenen innovativen Charakter enthält. \\

\textbf{Innovationscharakter} \\

Dieses Projekt kann als innovativ oder außergewöhnlich angesehen werden, da schwarze Löcher für einen Großteil der
Bevölkerung ein sehr unbekanntes Thema sind und eher einschüchternd wirken, wenn man die komplexe Mathematik dahinter
betrachtet.
Außerdem geht es um besonders abstrakte Konzepte, da bis heute erst zwei echte Bilder eines schwarzen Loches angefertigt
werden konnten~\autocite{Article:EHT_Imaging_M87,Article:EHT_Imaging_SgrA}.
Demnach ist das Projekt innovativ, weil es sich mit dieser komplizierten Thematik auseinandersetzt.
Darüber hinaus wird auch versucht, anderen Menschen ein Verständnis für das Thema zu vermitteln, indem eine
interaktive Applikation entwickelt wird, die es ermöglicht, spielerisch und experimentell die komplexen
Vorgänge in der Nähe eines schwarzen Loches zu erleben. \\

%\textbf{Rechtslage} \\



\textbf{Einzigartigkeit} \\

Es gibt ohnehin wenige Projekte, die als Ziel die Visualisierung schwarzer Löcher verfolgen.
Man könnte diese an einer Hand abzählen.
Jedoch ist bei diesen Projekten nie eine Benutzerinteraktion vorgesehen, was die Komplexität der technischen Umsetzung
nicht schmälert, aber auch keine Endanwendung finden wird.
Genau in dieser Lücke findet sich das MoleHole Projekt wieder, indem es die Echtzeitsimulation schwarzer Löcher mit einer
interaktiven Benutzeroberfläche kombiniert. \\

\subsubsection{Realisierung} \label{subsubsec:implementation} \\

Diese Sektion befasst sich mit der Implementierung der aktuellen Version der Applikation und soll
diese Schritt für Schritt beschreiben. \\

\textbf{Technologiestack} \\

Die gesamte Applikation wurde in C++ entwickelt.
Die einzige Programmiersprache, die zusätzlich verwendet wurde, ist GLSL für die Shader-Programmierung.
Die Programmiersprachen sind also sehr einfach gehalten.
Im Gegensatz dazu werden einige Bibliotheken verwendet, um nicht alles selbst entwickeln zu müssen.
Alle verwendeten Bibliotheken sind:
\begin{itemize}
    \item OpenGL Libraries: GLAD, GLFW, GLM~\autocite{GitHub:Glad,GitHub:Glfw,GitHub:Glm}
    \item Dear ImGui~\autocite{GitHub:ImGui}
    \item spdlog~\autocite{GitHub:Spdlog}
    \item PhysX~\autocite{GitHub:PhysX}
    \item stb\_image~\autocite{GitHub:Stb}
    \item ImGuizmo~\autocite{GitHub:ImGuizmo}
    \item tinygltf~\autocite{GitHub:Tinygltf}
    \item ImGui-Node-Editor~\autocite{GitHub:ImGuiNodeEditor}
    \item yaml-cpp~\autocite{GitHub:Yaml-cpp}
    \item nativefiledialog-extended~\autocite{GitHub:Nativefiledialog}
\end{itemize}

Zusätzlich zu diesen Technologien, gibt es auch einige Tools und Programme, die für die Entwicklung
verwendet wurden.
Dabei wurden CLion und YouTrack von JetBrains und Git als Versionskontrolle eingesetzt. \\

\textbf{Rendering Techniken} \\

Die wichtigsten Rendering Techniken, die in der Applikation verwendet werden, sind Ray Tracing und
Ray Marching.
Beim Ray Tracing geht von der Kamera aus bis zu einem Schnittpunkt mit einem Objekt ein Lichtstrahl
aus und wird dort reflektiert und zur Lichtquelle zurückverfolgt.
Dies wird für eine große Anzahl an Schnittpunkten durchgeführt, bis schlussendlich das ganze Objekt sichtbar ist.
Dies wird dann am Bildschirm dargestellt.
Beim Ray Marching nutzt man auch Lichtstrahlen, die von der Kamera ausgehen, aber man geht hier Schritt für
Schritt und rechnet nicht sofort den Schnittpunkt mit einem Objekt aus.
Grundsätzlich ist diese Methode also rechenintensiver, aber sie ermöglicht es, gekrümmte Lichtstrahlen abzubilden.
Diese gekrümmten Lichtstrahlen sind dafür verantwortlich, dass ein schwarzes Loch aussieht, wie es aussieht.
Die Lichtstrahlen werden durch die Gravitation abgelenkt und dieser sogenannte Gravitationslinseneffekt
wird durch das Ray Marching in der Applikation simuliert.
Da die Rechenleistung eines normalen Computers für das Ray Marching nicht ausreicht, muss man einen Kompromiss eingehen.
Man verzichtet auf die exakte Darstellung der Objekte und beschäftigt sich nur mit der Simulation von Lichtstrahlen,
die sehr nahe an einer großen Masse vorbeigehen.
Dadurch wird gewährleistet, dass die gewünschten optischen Effekte zu sehen sind, jedoch auch nicht zu viel Genauigkeit
verloren geht, da die Lichtstrahlen in größerer Entfernung nicht mehr wirklich sichtbar abgelenkt werden.
Für die Software werden also beide Rendering Techniken eingesetzt: Ray Tracing um Lichtstrahlen bis zu einer gewissen
Entfernung zu schwarzen Löchern zu simulieren und Ray Marching um die Lichtstrahlen in der Nähe von schwarzen
Löchern zu berechnen. \\

%Antialiasing (FXAA);
%Bloom Effekt;
%Third-Person-Kamera;
%irst-Person-Kamera; \\

\textbf{Physikalische Effekte} \\

Die physikalischen Effekte, die bis zum jetzigen Stand in der Software realisiert worden sind,
bilden die Grundlage für die Simulation schwarzer Löcher.

Der wichtigste Effekt betrifft den \textit{Gravitationslinseneffekt}, der sich durch die starke
Raumzeitkrümmung in der Nähe eines schwarzen Loches ergibt.
Hierbei werden Lichtstrahlen in der Nähe einer großen Masse stärker abgelenkt als in größerer Entfernung.
Dies entspricht dann einem ähnlichen Effekt wie bei einer herkömmlichen Konvexlinse.

Die Theorie, die vorerst zur Simulation der Gravitation verwendet wird, ist die Gravitationstheorie
von Isaac Newton.
Diese Theorie ist schon sehr alt, funktioniert aber immer gut für die Berechnung der Anziehungskräfte
zwischen Objekten mit geringen Massen.
$G$ sei die Gravitationskonstante, $M$ und $m$ die Massen zweier Objekte und $\vec{r}$ der Abstand
zwischen den Zentren der beiden Objekte.
Die Gravitationskraft $\vec{F}_G$ berechnet sich dann durch~\autocite{Book:Principia_Mathematica}

\begin{equation}
    \vec{F}_G = \frac{GMm}{\vec{r}^2}.
    \label{eq:gravitational_force}
\end{equation}

Für die Visualisierung der Raumzeitkrümmung wird die Schwarzschildmetrik verwendet.
Diese wird zur Berechnung der Krümmung in der Nähe einer Punktmasse verwendet.
Es seien die Distanz zwischen zwei Ereignissen $ds$, der Schwarzschildradius $R_S$, die Entfernung
vom Zentrum der Masse $R$ sowie die Dimensionen der Raumzeit $dt$, $dr$ und $d\Omega$ gegeben.
Dann beschreibt

\begin{equation}
    ds^2 = -\Bigg(1 - \frac{R_S}{R}\Bigg)dt^2 + \frac{1}{(1 - \frac{R_S}{R})}dr^2 + d\Omega^2
    \label{eq:schwarzschild_metric}
\end{equation}

die Raumzeitkrümmung in der Nähe einer
Punktmasse~\autocite{Book:Gravitation,Book:General_Relativity,Book:Black_Holes_bc}. \\

%Physikalische Effekte:

%Dopplereffekt;
%Gravitative Rotverschiebung;
%Akkretionsscheibe;
%ISCO;
%Photonensphäre;
%Schwarzkörperstrahlung;
%Hertzsprung-Russell-Diagramm; \\

\textbf{Forschungsmethoden} \\

Als Forschungsmethode wurde sich auf die Empirie gestützt, um die Applikation zu verbessern.
Es wurde ein Problem identifiziert und danach an einer vermeintlichen Lösung gearbeitet.
Nachdem man eine Lösung auf dem Papier entwickelt hatte, wurde diese in der Applikation implementiert und ausprobiert.
Durch die Überprüfung der neuen Implementierung konnte eine Entscheidung getroffen werden, ob die Lösung
zielführend war oder nicht.

Außerdem war die Recherche ein wichtiges Mittel, um an nötige Informationen zu gelangen.
Hierfür wurden wissenschaftliche Artikel, Bücher und Videos wie zum
Beispiel~\autocite{Book:General_Relativity,Book:Black_Holes_bc,Video:Coding_Adventure_Clouds}
verwendet. \\

\subsubsection{Ergebnisse} \label{subsubsec:results} \\

Diese Sektion beschäftigt sich mit dem Fortschritt, der bis heute erreicht werden konnte.
Dabei werden einige Bilder zu sehen sein und anhand einiger Zahlen der Umfang beschrieben. \\

\textbf{Ergebnisse in visueller Darstellung} \\

\begin{figure}[b]
    \centering
    \begin{subfigure}{0.4\textwidth}
        \centering
        \includegraphics[width=\linewidth]{img/02_Black_Hole_Rendering_White}
        \caption{Rendering eines schwarzen Loches mit Akkretionsscheibe und Gravitationslinseneffekt.}
        \label{fig:black_hole_rendering_white}
    \end{subfigure}
    \begin{subfigure}{0.4\textwidth}
        \centering
        \includegraphics[width=\linewidth]{img/Export-JI-07}
        \caption{Rendering eines schwarzen Loches mit Akkretionsscheibe und Gravitationslinseneffekt.}
        \label{fig:black_hole_rendering_black}
    \end{subfigure}
    \caption{Rendering eines schwarzen Loches mit Akkretionsscheibe und Gravitationslinseneffekt.}
    \label{fig:black_hole_rendering}
\end{figure}

Die beiden Rendering-Ergebnisse in Abbildung~\ref{fig:black_hole_rendering} zeigen jeweils ein
einziges schwarzes Loch.
Bei dem linken Bild in Abbildung~\ref{fig:black_hole_rendering_white} ist die Akkretionsscheibe als weiß zu
erkennen.
Das kommt daher, da dieses Rendering in einem früheren Stadium des Projektes aufgenommen worden ist und
die Temperaturskalierung nicht den realen Verhältnissen entsprochen hat.
Im rechten Bild in Abbildung~\ref{fig:black_hole_rendering_black} kann man eine genauere Verteilung der
Temperatur in der Akkretionsscheibe erkennen, welche sich in Abhängigkeit der Entfernung zum Zentrum
des schwarzen Loches ändert.

Man nehme an, dass $G$ die Gravitationskonstante, $\sigma$ die Stefan-Boltzmann-Konstante,
$M$ die Masse des schwarzen Loches, $\dot{M}$ die Akkretionsrate, $R_i$ der innere Radius der Akkretionsscheibe
und $R$ der aktuelle Radius in der Akkretionsscheibe ist.
Die Temperaturverteilung $T(R)$ in der Akkretionsscheibe lässt sich dann durch

\begin{equation}
    T(R) = \Bigg[ \frac{3GM\dot{M}}{8\pi \sigma R^3} \Bigg( 1 - \sqrt{\frac{R_i}{R}}\Bigg) \Bigg]^{\frac{1}{4}}\label
    {eq:temp_scaling_accretion_disk}
\end{equation}

beschreiben~\autocite{Article:Temp_Accretion_Disk}.

\begin{figure}[b]
    \centering
    \begin{subfigure}{0.45\textwidth}
        \centering
        \includegraphics[width=\linewidth]{img/Export-JI-03}
        \caption{First-Person-Perspektive}
        \label{fig:first_person_perspective}
    \end{subfigure}
    \begin{subfigure}{0.45\textwidth}
        \centering
        \includegraphics[width=\linewidth]{img/Export-JI-04}
        \caption{Third-Person-Perspektive}
        \label{fig:third_person_perspective}
    \end{subfigure}
    \caption{Vergleich der First-Person- und Third-Person-Perspektive.}
    \label{fig:camera_perspectives}
\end{figure}

In der Software wurden zwei verschiedene Kameraperspektiven umgesetzt: die First-Person-Perspektive und
die Third-Person-Perspektive.
In Abbildung~\ref{fig:camera_perspectives} kann man die beiden im Vergleich betrachten.
Für die Third-Person-Perspektive kann man ein Objekt auswählen, das man dann steuern kann.
Dieses Objekt wird mit einem zusätzlichen Würfel, welcher eine Art Crosshair darstellen soll, in der
Applikation gerendert. \\

%Dreifachsternsystem:

%Visualisierung Raumzeitkrümmung:

%Performance: \\

\textbf{Erreichung der Ziele} \\

Das Projektteam setzt sich alle Ziele am Anfang eines sogenannten Sprints~\autocite{Website:Scrum} und schreibt
diese in der Projektmanagement-Software YouTrack~\autocite{Website:YouTrack} nieder.
Dort werden allgemein die Ziele formuliert, mit einer Priorität versehen, der Arbeitsaufwand anhand von Story Points
geschätzt und ein Verantwortlicher festgelegt.
Um zu überprüfen, ob das Projektteam die Ziele erfüllt, werden diese in der Applikation von den Mitgliedern des
Projektteams ausprobiert.
Da dies keine besonders zuverlässige Methode zur Bestimmung der erfüllten Ziele ist, gibt es am Ende eines jeden
Sprints ein Sprint-Review Meeting, bei dem das Team ihre Ergebnisse dem Projektbetreuer sowie einem ausgewählten Kunden
(ein Mitschüler) vorstellt und den Sprint somit abschließt~\autocite{Website:Scrum}.
Danach werden erneut Ziele für die nächste Iteration überlegt. \\

\subsubsection{Entwicklungspotenzial} \label{subsubsec:potential} \\

Es steckt noch einiges an Potenzial in diesem Projekt und das Team plant nicht, dieses einfach wegzuwerfen und
nicht zu nutzen.
Demnach werden in dieser Sektion alle potenziellen Erweiterungen des Projektes aufgezählt. \\

\textbf{Gesellschaftliche Auswirkungen} \\

Wie bereits in~\ref{subsec:project_origin_and_planning} erwähnt ist das Gesamtziel des Projektes eine verbesserte
Wissenschaftskommunikation durch eine Applikation mit der man experimentell und spielerisch komplexe Vorgänge
unseres Universums erleben und erlernen kann.
Das Produkt ist so konzipiert, dass es die Möglichkeit gibt, eigene Simulationen zu erstellen und in diesen zu experimentieren,
aber auch über Tooltips und Dokumentation Genaueres über die simulierten physikalischen Effekte zu erfahren.
Weiters ist die Applikation als OpenSource-Software frei auf GitHub verfügbar und wird dies in Zukunft auch bleiben, was
den gesellschaftlichen Nutzen erheblich verbessert, weil es jeder, der einen PC besitzt, ausprobieren kann.
Insgesamt lässt sich feststellen, dass es dem Projektteam ein Anliegen ist, mit dieser Applikation die
Wissenschaftskommunikation zu verbessern und damit mehr Menschen den Zugang zur Wissenschaft zu erleichtern. \\

\textbf{Kooperationen} \\

Aktuell gibt es bei diesem Projekt keine bestehenden Kooperationen, jedoch würde das Projektteam in Zukunft welche anstreben.
Allerdings ist das Projekt noch sehr jung und das Team möchte zuerst wissen, wie sich die Ziele und das Produkt entwickeln. \\

\textbf{Releases} \\

Wenn die Entwicklung des Projektes weiterhin große Fortschritte erzielt, ist eine Veröffentlichung des Projektes
zum Beispiel auf Steam anzuvisieren.
Aktuell kann man das Projekt zwar schon benutzen, da es ein öffentliches GitHub Repository mit dem Code gibt, aber
es ist ein wenig technisches Wissen notwendig, um dieses zu verwenden.
Wenn die Applikation auf einer Vertriebsplattform wie Steam veröffentlicht werden kann, würde das den Zugang
zu der Software erleichtern.
Vorausschauend ist zu sagen, dass der Plan eines offiziellen Releases besteht, aber noch einige Zeit in der Zukunft liegt.
Das bedeutet, dass sich das Projektteam derzeit darum kümmert, die notwendigen Features hinzuzufügen, die dem Team noch wichtig sind
und danach die nötigen Schritte in Richtung eines Releases gehen wird. \\

\textbf{Technische Potenziale} \\

Das mit Abstand wichtigste physikalische Phänomen, das in der Applikation noch umgesetzt werden muss, ist die
\textit{Kerr-Metrik}, welche im Jahre 1963 vom neuseeländischen Physiker Roy Kerr als Lösung der Feldgleichungen aus der
\textit{allgemeinen Relativitätstheorie (ART)} aufgestellt worden ist~\autocite{Book:General_Relativity}.
Hierfür wird allerdings Wissen aus der ART benötigt, das sich das Projektteam noch nicht angeeignet hat.
Deshalb wird es noch einige Zeit dauern, bis dieses Feature verwirklicht werden kann.
Nichtsdestotrotz beschäftigt sich das Team mit diesem Effekt, da dieser extrem relevant für eine akkurate Darstellung
eines schwarzen Loches ist.
Mit der Kerr-Metrik folgen einige andere wichtige Phänomene aus der ART, die als mögliche Ziele des Projektes
im Hinterkopf behalten werden sollen, jedoch erst nach Umsetzung der Kerr-Metrik implementiert werden können.
Diese Effekte sind das Frame-Dragging, die Zeitdilatation, die Längenkontraktion sowie die Gravitation im Allgemeinen.
Vor allem Zeitdilatation und Längenkontraktion sind wichtig für eine genaue Simulation schwarzer Löcher, da für Beobachter
außerhalb der unmittelbaren Umgebung eines schwarzen Loches Objekte nahe dem Ereignishorizont sich immer langsamer
bewegen und letztendlich verblassen~\autocite{Book:Gravitation}.
Hierbei muss natürlich auch die Gravitation implementiert werden, die der wichtigste Bestandteil der zukünftigen Ziele
sein wird.
Es soll dabei die aktuell verwendete Simulation der Gravitation nach der Gravitationstheorie von Isaac
Newton~\autocite{Book:Principia_Mathematica} durch die Gravitationstheorie von Albert Einstein, die ART, ersetzt werden.
Mit dieser Theorie wäre es möglich, Objekte, die sich sehr nahe an großen Massen befinden, genauestens zu simulieren.
Generell lässt sich sagen, dass die Theorie der Gravitation von Isaac Newton in der Nähe von schwarzen Löchern nicht
funktionieren wird, da die Raumzeit hier besonders stark gekrümmt ist~\autocite{Book:Gravitation,Book:General_Relativity}.
Aus der Gravitationstheorie lässt sich ein weiterer Effekt ableiten, der für eine genaue Simulation von Objekten in der
Nähe großer Massen von Relevanz ist: die Gezeitenkraft~\autocite{Paper:Tidal_forces}.
Mit der Implementierung dieser Kraft wäre es möglich, Himmelskörper wie zum Beispiel Millers Planet aus dem Film
Interstellar~\autocite{Film:Interstellar,Book:Science_of_Interstellar} physikalisch möglichst genau zu simulieren.
Aus technischer Perspektive soll die visuelle Darstellung der Objekte auch noch verbessert werden.
Hierfür wird das \textit{Volumetric Cloud Rendering} verwendet werden, um die Akkretionsscheibe eines schwarzen Loches
detailreicher darzustellen.
Diese Rendering-Technik kümmert sich um die Belichtung wolkenartiger Gebilde, da Licht in Wolken gestreut und abgelenkt wird,
sodass nicht genau das Licht den Beobachter erreicht, das ursprünglich in seine Richtung ausgesandt worden
ist~\autocite{Video:Coding_Adventure_Clouds,Paper:Volumetric_Cloud_Rendering}.
Außerdem wird eine Partikel Simulation angestrebt, sodass man sehen kann, wie sich über die Zeit eine Akkretionsscheibe um
ein schwarzes Loch bildet.
Abgesehen von der technischen Simulation, ist es auch das Ziel, Wissenschaft zu kommunizieren.
Daher ist auch geplant, Erklärungen und Beschreibungen mittels Tooltips und detaillierte Berichte in einem Menü unterzubringen.
Diese Erklärungen sollen auch nicht auf Mathematik verzichten, sollen diese aber verständlich erklären und wenn nötig auf externe
Ressourcen verweisen.
Somit wäre garantiert, dass interessierte Menschen relativ einfach an die nötigen Ressourcen gelangen können, die notwendig sind, um
komplexe physikalische Konzepte zu verstehen. \\

\textbf{Wirtschaftliches Potenzial} \\

Das Projektteam plant nicht, aus dem Projekt Gewinne auszuschöpfen.
Dies kommt daher, da Wissenschaft für alle zugänglich sein sollte und man den Zugang zu der Applikation mit einem Kaufpreis
einschränken würde.
Jedoch handelt es sich immer noch um eine einzigartige Software, welche theoretisch auch ein Premium-Abonnement erhalten könnte, um
für wissenschaftliche Zwecke an Hochschulen verwendet werden zu können.
Dieses würde Features wie zum Beispiel detaillierte Erklärungen zu physikalischen Phänomenen oder extrem rechenintensive Simulationen über einen
externen Server beinhalten. \\