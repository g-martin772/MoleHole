\textbf{Zusammenarbeit im Team:}

Die Zusammenarbeit im Team verlief durchwegs positiv.
Beide Mitglieder brachten ihre individuellen Stärken ein und ergänzten sich dadurch ausgezeichnet.
Es wurden viele Probleme gemeinsam besprochen und gelöst.
Auch die Zusammenarbeit mit dem Projektbetreuer war stets konstruktiv und hilfreich.
Er gibt dem Entwicklungsteam regelmäßig Feedback, welches zur Verbesserung des Projektes beiträgt.

\textbf{Kompetenzen im Team:}

Gabriel hat bereits Erfahrung in der Entwicklung von Applikationen im Bereich Computergrafik mit C++ und OpenGL\@.
Dies war besonders hilfreich zu Beginn des Projektes, da er das Grundgerüst der Applikation schnell aufsetzen konnte.
Er ist auch in den Bereichen Software Architektur und Performanceoptimierung sehr kompetent.

Leonhard bringt viel Wissen im Bereich der theoretischen Physik mit,
welches für die korrekte Umsetzung der physikalischen Effekte essenziell ist.
Zudem ist auch er ein talentierter Programmierer mit der Fähigkeit, sich schnell in neue Technologien einzuarbeiten.

Beide Mitglieder verfügen über viel Allgemeinwissen im Bereich moderner Softwareentwicklungsmethoden,
was die Organisation und Planung des Projektes erleichtert hat.

Viele der Grundkenntnisse der Physik und Informatik wurden im Rahmen der Ausbildung an der HTL Krems erworben.
Vertiefende Kenntnisse in den Bereichen Computergrafik,
Softwareentwicklung und theoretische Physik wurden durch eigenständige Recherche
und das Studium von Fachliteratur erlangt.

\textbf{Aufgabenverteilung:}

Die Aufgabenverteilung im Team erfolgte größtenteils nach den individuellen Stärken und Interessen der Mitglieder.
Gabriel übernahm hauptsächlich die Implementierung des Applikationsgerüsts,
die Integration von OpenGL und die Optimierung der Rendering-Performance.
Leonhard konzentrierte sich auf die Implementierung der physikalischen Modelle,
die Berechnung der Raumzeitkrümmung und die Simulation der Interaktionen zwischen Objekten.
Beide Mitglieder arbeiten jedoch eng zusammen und unterstützten sich gegenseitig bei der Lösung von Problemen.

In vielen Bereichen wie UI Design, Rendering und Research arbeiten beide Mitglieder gemeinsam,
um die bestmöglichen Ergebnisse zu erzielen.

\textbf{Schulische Projektbetreuung:}

Der Projektbetreuer, Jürgen Katzenschlager,
hatte zwar keine tiefgehenden Kenntnisse im Bereich der Computergrafik oder theoretischen Physik,
konnte aber durch seine Erfahrung in der Projektleitung und Softwareentwicklung wertvolle Ratschläge geben.
Er half dem Team,
den Fokus auf die wichtigsten Ziele zu legen, und unterstützte bei der Planung und Organisation des Projektes.
Seine regelmäßigen Reviews und Feedback-Sessions trugen dazu bei,
dass das Projekt auf Kurs blieb und die Qualität der Arbeit hoch war.

\textbf{Koordination:}

Die Umsetzung des Projektes erfolgt auf der Basis agiler Methoden mit 3 Wochen Sprints.
Wöchentliche Status-Meetings werden abgehalten, um den Fortschritt zu besprechen und eventuelle Probleme zu lösen.
Am Ende jedes Sprints findet ein Sprint-Review statt, bei dem die erreichten Ziele präsentiert und bewertet werden.
Die Aufgaben werden in einem Kanban-Board in YouTrack verwaltet,
wo auch der Fortschritt und die Prioritäten der einzelnen Aufgaben festgehalten werden.

Die Entwicklung verlief bisher größtenteils planmäßig,
obwohl es natürlich auch einige Herausforderungen gab,
besonders im Rendering und Realtime Performance Bereich,
welche zu leichten Verzögerungen führten.
An wichtigen Entscheidungspunkten fand sich das Team in einem Brainstorming-Meeting zusammen,
um die bestmögliche Lösung zu erarbeiten, welche anschließend erfolgreich umgesetzt wurde.

Die offene Kommunikation, das gegenseitige Verständnis und die gegenseitige Unterstützung
haben dazu beigetragen, dass das Team harmonisch zusammenarbeitet.
Bisher sind keine Konflikte im Team aufgetreten.

\textbf{Zeitaufwand und Kosten:}

Zum Stand des 19.1.2026 betrug der gesamte Zeitaufwand für das Projekt etwa 250 Stunden.
Dies umfasst sowohl die Entwicklungszeit als auch die Zeit für Recherche, Planung und Meetings.
Die Kosten für das Projekt waren minimal, da alle benötigten Werkzeuge und Ressourcen kostenlos verfügbar waren.

