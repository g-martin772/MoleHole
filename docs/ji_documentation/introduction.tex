%! Author = leonh
%! Date = 31.12.2025

Die Physik hinter schwarzen Löchern besteht aus komplizierten Theorien und Beobachtungen, die
außergewöhnliche Techniken erfordern~\autocite{Book:General_Relativity,Article:EHT_Imaging_SgrA,Article:EHT_Imaging_M87}.
Obgleich es schwierig ist, die Mathematik und die Ideen hinter diesen Konzepten zu verstehen, ist es möglich, ein
allgemeines Verständnis dieser Objekte aufzubauen.
Dafür setzt das MoleHole Projekt auf eine interaktive Oberfläche, die es Benutzern ermöglicht, schwarze Löcher und deren
Auswirkungen auf die Umgebung darzustellen und verschiedene Szenarien auszuprobieren.
Um diese unterschiedlichen Darstellungen zu erreichen, kann man Effekte wie den
\textit{Gravitationslinseneffekt}, die \textit{Raumzeitkrümmung}, die \textit{Akkretionsscheibe},
den \textit{Dopplereffekt} und die \textit{Gravitation} beobachten.
Dieses Dokument schildert die wichtigsten Fakten und Erkenntnisse aus
dem Projektverlauf.
Dabei wird auf die Idee der Simulation schwarzer Löcher, die anfängliche Planung und
das Projektmanagement eingegangen.
Weiters wird beschrieben, wie dieses Projekt einen bleibenden Eindruck in der Gesellschaft
hinterlassen kann und wie es die Wissenschaftskommunikation verbessern soll.
Danach werden einige technische Details erklärt, die im Hintergrund für die Visualisierung
und Simulation der schwarzen Löcher verantwortlich sind.
Es werden auch einige Grafiken zu sehen sein, die verdeutlichen sollen, wie die Applikation, die das
Ergebnis des Projekts bildet, aussieht.
Damit kann man erkennen, welche verschiedenen Möglichkeiten die Software bietet.
Ein wichtiger Teil ist außerdem der Ausblick und das Potenzial, das in diesem Projekt steckt.
Zum Abschluss wird ein Bericht über den Projektverlauf vom Projektkoordinator,
Gabriel Martin, verfasst.
