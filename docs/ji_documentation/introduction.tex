%! Author = leonh
%! Date = 31.12.2025

Schwarze Löcher sind komplexe Objekte.
Jedoch kann diese jeder verstehen, wenn er möchte.
Dies ist mit spielerischem Erlernen und Ausprobieren möglich.
Das ist das Konzept des MoleHole Projekts.
Dieses Dokument schildert die wichtigsten Fakten und Erkenntnisse aus
dem Projektverlauf.
Dabei wird auf die Idee der Simulation schwarzer Löcher, die anfängliche Planung und
das Projektmanagement eingegangen.
Weiters wird beschrieben, wie dieses Projekt einen bleibenden Eindruck in der Gesellschaft
hinterlassen kann und wie es die Wissenschaftskommunikation verbessern soll.
Danach werden einige technische Details erklärt, die im Hintergrund für die Visualisierung
und Simulation der schwarzen Löcher verantwortlich sind.
Es werden auch einige Grafiken zu sehen sein, die verdeutlichen sollen, wie die Applikation, die das
Ergebnis des Projekts bildet, aussieht.
Damit kann man erkennen, welche verschiedenen Möglichkeiten die Software bietet.
Ein wichtiger Teil ist außerdem der Ausblick und das Potenzial, das in diesem Projekt steckt.
Zum Abschluss wird ein Bericht über den Projektverlauf vom Projektkoordinator,
Gabriel Martin, verfasst.
