%! Author = leonh
%! Date = 31.12.2025

\subsection{Projektentstehung und -planung} \label{subsec:project_origin_and_planning}

Der folgende Abschnitt soll sich mit der Phase vor dem Beginn des eigentlichen Projektes und der darauffolgenden
Planung auseinandersetzen. \\

\begin{figure}[b]
    \centering
    \includegraphics[width=0.7\textwidth]{img/01_Planning_Notes}
    \caption{Originale Notizen bei der Aufstellung der Projektziele}
    \label{fig:original_notes_project_goals}
\end{figure}

\subsubsection{Projektentstehung} \label{subsubsec:project_origin} \\

In dieser Sektion werden die Projektidee und das allgemeine Ziel des Projektes behandelt.
Dabei wird auf die äußeren Umstände sowie die eigenen Ideen eingegangen. \\

\textbf{Projektidee} \\

Im Unterrichtsgegenstand \textit{Informationstechnische-Projekte-Labor (ITPL)} an der
HTL Krems~\autocite{Website:HTL_Krems} hatte das Projektteam die Aufgabe,
ein Projekt nach dessen eigenen Vorstellungen zu planen und durchzuführen.
Die beiden Mitglieder des Projektteams waren frustriert mit dem Verlauf deren früherer Projekte
und wollten etwas Neuartiges entwickeln.
Daraus entsprang die Idee, schwarze Löcher virtuell und visuell darzustellen und in Echtzeit zu simulieren.
Im Zuge der Einigung auf ein Gesamtziel versuchte das Projektteam, diese Vorstellung in einzelne
Teile zu gliedern.
In Abbildung~\ref{fig:original_notes_project_goals} sind die originalen Notizen des Projektteams zu
sehen, die bei der Überlegung der Projektziele entstanden sind.
Diese Ziele stellen die übergeordneten Gesamtausmaße des Projektes dar, welche das Team erreichen
möchte.
Im Zuge der besseren Lesbarkeit und genaueren Dokumentation werden diese sogenannten funktionalen
Anforderungen nochmals in höherer Genauigkeit aufgelistet:
\begin{itemize}
    \item   \textbf{Basis mit Dear ImGui:} Als Basis wird eine \textit{C++ Applikation} aufgesetzt, welche die
            \textit{Library} Dear ImGui nutzt~\autocite{GitHub:ImGui}, um dem Endnutzer die Möglichkeit zu bieten,
            verschiedenste \textit{Parameter} einzustellen, als auch sich im drei dimensionalen Raum zu bewegen.
    \item   \textbf{Visualisierung:} Zum besseren Verständnis schwarzer Löcher trägt vor allem die Visualisierung
            bei.
            Deshalb sollen der \textit{Ereignishorizont}, die \textit{Photonensphäre}, die Akkretionsscheibe und die
            Raumzeitkrümmung dargestellt werden.
            Zudem sollen auch noch verschiedenste Parameter eines schwarzen Loches angepasst werden können.
            Das sind die Masse $m$, Position $x^i$, Geschwindigkeit $v^i$ sowie der \textit{Spin-Parameter} $a$.
            Diese Punkte sollen mithilfe des \textit{Shader-Frameworks OpenGL} erfüllt werden.
    \item   \textbf{Simulation:} Um über die reine Darstellung hinaus auch die unvorstellbaren Kräfte
            schwarzer Löcher zu erkennen, soll es auch die Möglichkeit geben, eine Simulation zu konfigurieren
            und danach anzusehen.
            In dieser soll es möglich sein, Phänomene wie die Anziehungskräfte zwischen verschiedenen Objekten
            zu erfahren.
            Außerdem wird die Möglichkeit, andere Objekte wie z.B.\ Sterne, Planeten und Raumschiffe einzufügen und
            mit dem schwarzen Loch interagieren zu lassen, einen Teil der Implementierung ausmachen.
    \item   \textbf{Physikalische Korrektheit:} Damit ein Endnutzer auch lernen kann, wie schwarze Löcher in der
            realen Welt funktionieren, sollen möglichst genaue Theorien über schwarze Löcher angewendet werden.
            Zu diesen gehören z.B.\ die \textit{Kerr-Metrik} zur Berechnung der Raumzeitkrümmung in der Nähe rotierender
            schwarzer Löcher oder der Dopplereffekt zur Veränderung des ausgesandten Lichts.
\end{itemize} \\

\textbf{Allgemeines Ziel} \\

Weiters bildete sich über diese formulierten funktionalen Anforderungen auch ein allgemeines Ziel, das
lautet:
\begin{center}
    \textbf{Die komplizierte Physik unseres Kosmos sollte jedem Menschen so einfach und gut verständlich
    wie möglich präsentiert werden!}
\end{center}
Um den Worten des Astronomen Florian Freistetter gerecht zu werden~\autocite{Video:Podcast_Science_Communication},
soll den Menschen Wissenschaft beigebracht werden, indem man sie auf eine Reise einlädt, die Komplexität
verschiedenster Theorien weglässt und vor allem ein Vertrauen aufbaut.
Dieses Projekt soll das umsetzen und damit eine Hilfe in der Kommunikation von Wissenschaft mit der allgemeinen
Bevölkerung darstellen. \\

\textbf{These} \\

Es war nicht nötig, für dieses Projekt eine These zu formulieren, da es sich nicht um eine herkömmliche
wissenschaftliche Arbeit handelt, bei der eine Hypothese formuliert wird und im weiteren Verlauf der Arbeit
mithilfe wissenschaftlicher Methoden bestätigt oder wieder verworfen wird.
Dieses Projekt soll vielmehr ein Produkt hervorbringen, dass es den Menschen ermöglicht, auf einfache Art und
Weise Wissenschaft zu erfahren.
In diesem Projekt werden bei Erfüllung der Projektziele große Teile der Errungenschaften der Astronomie und
Kosmologie der letzten 150 Jahre umgesetzt und versucht, dem Endnutzer möglichst verständlich zu präsentieren. \\

\subsubsection{Recherche und Planung} \label{subsubsec:planning} \\

In dieser Sektion wird die Recherche und Planung des Projektes beschrieben.
Es wird dafür auf die verschiedenen Quellen eingegangen sowie die Art der Planung erklärt. \\

\textbf{Informationsgewinn} \\

Die Quellen für die Recherche und den Informationsgewinn sind so zahlreich, dass sie nicht in einer Liste aufgeführt
werden können, ohne den Rahmen dieses Dokumentes zu sprengen oder die Lesbarkeit zu beeinträchtigen.
Deshalb werden hier nur einige Beispiele aufgeführt werden.
Als wichtigste Ressourcen zählen einige Projekte, die als Inspiration für die Arbeit dienen.
Vor allem die Programmierung dieser Projekte war von großer Relevanz, da sie bei der Implementierung
verschiedener Effekte helfen.
Die Ressourcen sind:
\begin{itemize}
    \item   \textbf{Black Hole Raytracer - Ross Ning}: Dieses Projekt hat anfänglich als hauptsächliche Quelle für die Programmierung
            der Visualisierung schwarzer Löcher gedient.
            Es ist eine gute Simulation, auch wenn die Physik nicht perfekt umgesetzt
            ist~\autocite{Project:Black_Hole_Sim_rossning}.
    \item   \textbf{Black Hole Raytracer - Hydrogendeuteride}: In diesem Projekt sind vor allem der Dopplereffekt und die
            \textit{Schwarzkörperstrahlung} genau umgesetzt worden~\autocite{Project:Black_Hole_Sim_hdeu}.
    \item   \textbf{Thomas Kamminga}: Dieses Projekt ist vor allem aufgrund der korrekten physikalischen
            Implementierung eine gute Hilfe~\autocite{Project:Black_Hole_Sim_Kamminga}.
    \item   \textbf{Black Hole Raytracer - Eliot Han}: Die Dokumentation dieser Ressource war eine Hilfestellung, indem
            auf einer verlinkten Website die Lösung für die Implementierung des Gravitationslinseneffektes
            gut erklärt ist~\autocite{Project:Black_Hole_Sim_eliot,Website:Eliot_Han}.
    \item   \textbf{Andere}: Außerdem gibt es auch noch einige andere Projekte oder Beiträge, die von der
            Visualisierung schwarzer Löcher handeln, sie sind
            in~\autocite{Project:Black_Hole_Sim_20k,Project:Black_Hole_Sim_bhusie,Project:Black_Hole_Sim_starless,
            Reddit:Black_Hole_Sim_Burgao,Project:Black_Hole_Sim_sirxemic,Website:Starless} aufgezählt.
\end{itemize}
Viele dieser Projekte können auf GitHub betrachtet werden, während auch auf Reddit etwas zu finden war.
Auch Bücher waren eine ordentliche Hilfestellung bei der Implementierung.
Diese sind:
\begin{itemize}
    \item   \textbf{Spacetime and Geometry: An Introduction to General Relativity}: Dieses Buch ist 
            eine umfangreiche Einführung in die \textit{Allgemeine Relativitätstheorie (ART)} und erklärt
            die dafür notwendigen mathematischen Grundlagen~\autocite{Book:General_Relativity}.
    \item   \textbf{Schwarze Löcher: Der Schlüssel zum Verständnis des Universums}: In diesem Buch werden
            in populärwissenschaftlicher Weise bestimmte Phänomene und Theorien rund um schwarze Löcher
            erklärt~\autocite{Book:Black_Holes_bc}.
    \item   \textbf{Andere}: Es gibt natürlich auch noch viele andere Bücher zu dieser Thematik, allerdings
            wurden viele nicht für das Projekt verwendet oder es wurde nur wenig Wissen entnommen, wie
            aus~\autocite{Book:Gravitation,Book:Science_of_Interstellar}.
\end{itemize}
Insgesamt ist die Recherche für ein Projekt dieser Komplexität äußerst wichtig und dementsprechend viele
verschiedene Quellen wurden hierfür herangezogen.
Neben den oben genannten wichtigsten Ressourcen wurden auch einige wissenschaftliche Artikel, Dissertationen,
Videos und Webinhalte genutzt, um an das notwendige Wissen zu gelangen. \\

\textbf{Planung des Projektablaufes} \\

Die Planung des Projektes erfolgt agil, was nicht bedeutet, dass auf eine Planung im Vorfeld verzichtet
wurde.
Vielmehr wurde sich am Anfang die Zeit genommen, um die groben Ziele aufzustellen -
siehe~\ref{subsubsec:project_origin} - und Rahmenbedingen zu schaffen sowie das
Aufsetzen eines \textit{GitHub Repositories} und der \textit{Projektmanagementsoftware}
YouTrack~\autocite{Project:Black_Hole_Sim_MoleHole,Website:YouTrack}.
Darüber hinaus werden kleinere Aufgaben für jede Iteration eines sogenannten
\textit{Sprints}~\autocite{Website:Scrum} eingeteilt.
Die untergeordneten Projektziele werden im Laufe des Projektes angepasst und hinzugefügt oder gelöscht.
Es wird eine Mischung der agilen Projektplanungsframeworks \textit{Scrum}~\autocite{Wikipedia:Scrum} und
\textit{Kanban}~\autocite{Wikipedia:Kanban} - \textit{Scrumban}~\autocite{Wikipedia:Scrumban} - ausgeführt.
Dies wird insofern eingehalten, als das Rollen für die Mitglieder des Projektteams festgelegt wurden,
wöchentliche Meetings abgehalten werden, alle drei Wochen ein Product Increment erfolgt und alles über ein
Kanban-Board verwaltet wird.
Zudem werden mittels einer \textit{Knowledge Base} alle Meetings dokumentiert und andere wichtige Informationen
festgehalten.
Diese Art, ein Projekt zu planen und führen, ermöglicht eine flexible Arbeitsweise und macht besonders
im Falle dieses Projektes Sinn.
Dies kommt daher, dass die beiden Richtungen der Programmierung und der Physik miteinander kombiniert werden
müssen und dies einen ständigen Paradigmenwechsel sowie eine Überarbeitung alter Konzepte erfordert.
Da alle Beteiligten des Projektes großes Engagement zeigen, wird auch außerhalb des Unterrichts gearbeitet und
versucht, die aktuellen Ziele zu erfüllen. \\