%! Author = leonh
%! Date = 31.12.2025

\subsection{Projektentstehung und -planung} \label{subsec:project_origin_and_planning}

\begin{figure}[b]
    \centering
    \includegraphics[width=0.4\textwidth]{}
    \caption{Originale Notizen bei der Aufstellung der Projektziele}
    \label{fig:original_notes_project_goals}
\end{figure}

\textbf{Projektidee} \\

Die Projektidee entstammt einem spontanen Einfall.
Die beiden Mitglieder des Projektteams waren frustriert mit dem Verlauf deren früherer Projekte
und wollten etwas Neuartiges entwickeln.
Daraus entsprang die Idee, schwarze Löcher virtuell darzustellen und in Echtzeit zu simulieren.
Im Zuge der Einigung auf ein Gesamtziel versuchte das Projektteam, diese Vorstellung in einzelne
Teile zu gliedern.
In Abbildung~\ref{fig:original_notes_project_goals} sind die originalen Notizen des Projektteams zu
sehen, die bei der Überlegung der Projektziele entstanden sind.
Diese Ziele stellen die übergeordneten Gesamtausmaße des Projektes dar, welche das Team erreichen
möchte.
Im Zuge der besseren Lesbarkeit und genaueren Dokumentation werden diese sogenannten funktionalen
Anforderungen nochmals in höherer Genauigkeit aufgelistet:
\begin{itemize}
    \item   \textbf{Basis mit Dear ImGui:} Als Basis wird eine C++ Applikation aufgesetzt, welche die
            Library Dear ImGui nutzt, um dem Endnutzer die Möglichkeit zu bieten, verschiedenste Parameter
            einzustellen, als auch sich im drei dimensionalen Raum zu bewegen.
    \item   \textbf{Visualisierung:} Zum besseren Verständnis schwarzer Löcher trägt vor allem die Visualisierung
            bei.
            Deshalb sollen der Ereignishorizont, die Photonensphäre, die Akkretionsscheibe und die
            Raumzeitkrümmung dargestellt werden.
            Zudem sollen auch noch verschiedenste Parameter eines schwarzen Loches angepasst werden können.
            Das sind die Masse $m$, Position $x^i$, Geschwindigkeit $v^i$ sowie der Spin-Parameter $a$.
            Diese Punkte sollen mithilfe des Shader-Frameworks OpenGL erfüllt werden.
    \item   \textbf{Simulation:} Um über die reine Darstellung hinaus auch die unvorstellbaren Kräfte
            schwarzer Löcher zu erkennen, soll es auch die Möglichkeit geben, eine Simulation zu konfigurieren
            und danach anzusehen.
            Dabei soll es möglich sein, andere Objekte wie z.B.\ Sterne, Planeten und Raumschiffe einzufügen und
            mit dem schwarzen Loch interagieren zu lassen.
    \item   \textbf{Physikalische Korrektheit:} Damit ein Endnutzer auch lernen kann, wie schwarze Löcher in der
            realen Welt funktionieren, sollen möglichst genaue Theorien über schwarze Löcher angewendet werden.
            Zu diesen gehören z.B.\ die Kerr Metrik zur Berechnung der Raumzeitkrümmung in der Nähe rotierender
            schwarzer Löcher oder der Dopplereffekt zur Veränderung des ausgesandten Lichts.
\end{itemize} \\

\textbf{Allgemeines Ziel} \\

Weiters bildete sich über diese formulierten funktionalen Anforderungen auch ein allgemeines Ziel, das
lautet:
\begin{center}
    \textbf{Die komplizierte Physik unseres Kosmos sollte jedem Menschen so einfach und gut verständlich
    wie möglich präsentiert werden!}
\end{center}
Um den Worten des Astronomen Florian Freistetter gerecht zu werden~\autocite{Video:Podcast_Science_Communication},
soll den Menschen Wissenschaft beigebracht werden, indem man sie auf eine Reise einlädt, die Komplexität
verschiedenster Theorien weglässt und vor allem ein Vertrauen aufbaut.
Dieses Projekt soll das umsetzen und damit eine Hilfe in der Kommunikation von Wissenschaft mit der allgemeinen
Bevölkerung darstellen. \\

\textbf{These} \\

Es war nicht nötig, für dieses Projekt eine These zu formulieren, da es sich nicht um eine herkömmliche
wissenschaftliche Arbeit handelt, bei der eine Hypothese formuliert wird und im weiteren Verlauf der Arbeit
mithilfe wissenschaftlicher Methoden bestätigt oder wieder verworfen wird.
Dieses Projekt soll vielmehr ein Produkt hervorbringen, dass es den Menschen ermöglicht, auf einfache Art und
Weise Wissenschaft zu erfahren.
In diesem Projekt werden bei Erfüllung der Projektziele große Teile der Errungenschaften der Astronomie und
Kosmologie der letzten 150 Jahre umgesetzt und versucht, dem Endnutzer möglichst verständlich zu präsentieren. \\

\textbf{Informationsgewinn} \\

Die Quellen für die Recherche und den Informationsgewinn sind so zahlreich, dass sie nicht in einer Liste aufgeführt
werden können, ohne den Rahmen dieses Dokumentes zu sprengen oder die Lesbarkeit zu beeinträchtigen.
Deshalb werden hier nur einige Beispiele aufgeführt werden.
Als wichtigste Ressourcen zählen einige Projekte, die als Inspiration für die Arbeit dienen.
Sie sind in~\autocite{Project:Black_Hole_Sim_eliot, Project:Black_Hole_Sim_rossning, Project:Black_Hole_Sim_hdeu,
    Project:Black_Hole_Sim_Kamminga, Reddit:Black_Hole_Sim_Burgao} zu sehen.
Viele dieser Projekte können auf GitHub betrachtet werden, während auch auf Reddit einiges zu finden war.
Auch Bücher waren eine ordentliche Hilfestellung bei der Implementierung, wie~\autocite{Book:General_Relativity,
    Book:Black_Holes_bc} \\

\textbf{Planung des Projektablaufes} \\

Die Planung des Projektes erfolgt agil und es wurden dementsprechend keine Meilensteine vordefiniert.
Die untergeordneten Projektziele werden im Laufe des Projektes angepasst und hinzugefügt oder gelöscht.
Es wird eine Mischung der agilen Projektplanungsframeworks Scrum~\autocite{Wikipedia:Scrum} und Kanban~\autocite{
    Wikipedia:Kanban} - Scrumban~\autocite{Wikipedia:Scrumban} - ausgeführt.
Dies wird insofern eingehalten, als das Rollen für die Mitglieder des Projektteams festgelegt wurden,
wöchentliche Meetings abgehalten werden, alle drei Wochen ein Product Increment erfolgt und alles über ein
Kanban-Board verwaltet wird.
Zudem werden mittels einer Knowledge Base alle Meetings dokumentiert und andere wichtige Informationen
festgehalten.
Diese Art, ein Projekt zu planen und führen, ermöglicht eine flexible Arbeitsweise und macht besonders im Falle dieses
Projektes Sinn, da am Anfang den Mitgliedern des Projektteams nicht klar, wohin die Reise gehen soll.
Eigentlich war die Idee hinter dem Projekt eine Abwechslung in Bezug auf die restlichen Unterrichtsfächer an
der HTL Krems~\autocite{Website:HTL_Krems}.
Da alle Beteiligten des Projektes großes Engagement zeigen, wird auch außerhalb des Unterrichts gearbeitet und
versucht, die aktuellen Ziele zu erfüllen. \\